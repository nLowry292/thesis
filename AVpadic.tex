\externaldocument{intro}

\chapter{Abelian varieties over $p$-adically closed fields}

% Michael's comments on the first draft:
%The same issues come up for p-adic closed fields.  In Def. 3.0.1, what does \dim O/p mean?  I assume it means dimension as a vector space over the field with p elements, but you do not say this. What is a p-valued algebraic extension?  Do you assume it is unramified?  Because otherwise, it seems that \Q_p itself is not p-adically closed.  And why should we believe that any such fields exist other than \Q_p?  The language of valued fields is a staple of model theory, but it isn't well known to, for example, algebraic geometers or number theorists.  When you speak of an extension of a divisible group by Z with the lexicographic order, there are two possible lexicographic orders.  Which one do you mean?
%
%Higher dimensional formal group laws.  The vast majority of discussion of formal group laws concerns the 1-dimensional case.  You should say what g-dimensional FGLs are and how they arise from g-dimensional algebraic groups.  Serre is a good reference and can help you do this, but there is something wrong in what you wrote, I think, since I believe A^i/A^{i+1} should be a g-dimensional vector space.

Next, we apply the transfer principle to abelian varieties over $p$-adically closed fields. We begin with a detailed discussion of $p$-adically closed fields.

\section{$p$-adically closed fields}

The canonical reference for this material is \cite{formallyp-adic}, which begins with the statement ``The notion of a $p$-adically closed field axiomatizes the algebraic properties of the $p$-adic number field $\bQ_p$ in the same way as the notion of a real closed field axiomatizes the algebraic properties of the real number field $\bR$." Moreover, the theory of $p$-adically closed fields is developed in complete analogy with that of real closed fields: $p$-valued fields play the role of ordered fields (these are fields that possess valuations of a certain $p$-adic flavor to be discussed below); their maximal $p$-valued algebraic extensions are dubbed \emph{$p$-adically closed} fields; one can axiomatize the theory of $p$-adically closed fields in the language of valued fields $\mathcal{L}_{div}=\mathcal{L}_r\cup\{\mid\}$ and this theory is model complete; Hensel's lemma turns out to be the $p$-adic analogue of the intermediate value theorem for polynomials, while the analogue of the fact that every positive element has a square root over a real closed field reveals itself in a divisibility condition on the value group.

Let us begin with a definition.

\begin{definition}
Let $(K,v)$ be a valued field of characteristic zero with valuation ring $\mathcal{O}$ such that the residue field has characteristic $p>0$. $K$ is called a \emph{p-valued field} of $p$-rank $d$ if $\dim \mathcal{O}/p=d$, where the dimension is to be understood over the prime field $\bZ/p\bZ$.
\end{definition}

For the remainder of this thesis, the term ``$p$-valued field" is always used to refer to a $p$-valued field of rank 1.

\begin{example}
	\begin{enumerate}
		\item $\bQ_p$ with its $p$-valuation $v_p$ is a $p$-valued field. Of course, the restriction of $v_p$ to any subfield of $\bQ_p$ gives another $p$-valued field.
		\item Let $(K,v)$ be $p$-valued. We can define a $p$-valuation $w$ on the field of rational functions $K(t)$ that extends $v$; $w$ takes values in $\bZ\times v(K)$ ordered lexicographically, i.e. $(a,b)<(a',b')$ if $a<a'$ or $(a=a'$ and $b<b')$. For $f(t)\in K[t]$, we define $w(f):=(n,v(a_n))$, where $a_nt^n$ is the lowest-order term of $f$. We extend $w$ to $K(t)$ in the obvious way.
		\item We can extend $w$ from the previous example to the field of Puiseux series $K\langle\langle t\rangle\rangle$. For $f=\sum_{i=k}^{\infty} a_it^{i/N}$ and $a_k\neq 0$, we set $w(f):=(k/N, v(a_k))$, now taking values in $\bQ\times v(K)$.
	\end{enumerate}
\end{example}

Again, we are led naturally to consider maximal algebraic extensions of $p$-valued fields. 
\begin{definition}
Let $K$ be a $p$-valued field. $K$ is called \emph{$p$-adically closed} if $K$ does not admit any proper $p$-valued algebraic extensions.
\end{definition}

\begin{example}
	\begin{itemize}
		\item $\bQ_p$ is $p$-adically closed.
		\item The field of $p$-adic algebraic numbers $\bar{\bQ}\cap \bQ_p$ is $p$-adically closed.
		\item The field of Puiseux series $\bQ_p\langle\langle t\rangle\rangle$ is $p$-adically closed.
	\end{itemize}
\end{example}

A valued field $(K,v)$ can be realized as a $\mathcal{L}_{div}$-structure by interpreting $x\mid y$ iff $v(x)\leq v(y)$. As stated above, the theory of $p$-adically closed fields will turn out to be a model complete $\mathcal{L}_{div}$-theory, but the definition of a $p$-adically closed field given above \emph{a priori} cannot be captured in first-order logic. This seeming issue is resolved by the following

\begin{theorem}
A valued field $K$ is $p$-adically closed if and only if $K$ is Henselian and its value group is a $\bZ$-group, meaning $v(K)$ is an extension of a divisible group by $\bZ$, ordered lexicographically.
\end{theorem}
\begin{proof}
\cite[Theorem 3.1]{formallyp-adic}
\end{proof}

\begin{remark}
A $\bZ$-group is a model of Presburger arithmetic, i.e. an  ordered abelian group with the same theory (= set of true sentences) as $(\bZ,0,1,+,<)$.
\end{remark}

That $K$ is Henselian and that the value group has the desired divisibility property can indeed be captured in (infinitely many) $\mathcal{L}_{div}$-sentences. Thus we define $\catname{pCF}$ to be the $\mathcal{L}_{div}$-theory consisting of sentences corresponding to
\begin{itemize}
	\item the axioms of a valued field of characteristic zero,
	\item the residue field is (isomorphic to) $\bZ/p\bZ$,
	\item the field is Henselian,
	\item the value group is a $\bZ$-group.
\end{itemize}
TODO: add the actual sentences

As mentioned above, $\catname{pCF}$ is model complete:

\begin{theorem}
Let $K$ be $p$-adically closed. For any $\mathcal{L}_{div}$-sentence $\phi$, $K\models \phi$ iff $\bQ_p\models \phi$.
\end{theorem}
\begin{proof}
\cite[Theorem 5.1]{formallyp-adic}
\end{proof}

\section{Abelian varieties over $\bQ_p$}
The result in this section can be found, for instance, in \cite{clark2019there}. We reproduce the proof to illustrate it's $p$-adic analytic nature, once again highlighting Weil's philosophy of the Lefschetz principle.

TODO: g-dimension formal group law associated to g-dimensional algebraic group. apparently this can be found in serre

\begin{theorem}
\label{structure-of-padic-AVs}
Let $A/\bQ_p$ be an abelian variety of dimension $g$. Then $A(\bQ_p)[\tors]$ is finite and there is an isomorphism of topological groups $$A(\bQ_p)\isom \bZ_p^g\times A(\bQ_p)[\tors]$$
\end{theorem}
\begin{proof}
Associated to $A$ is a $g$-dimensional formal group law. We obtain a filtration by open subgroups $$A(\bQ_p)=A^0\supset A^1\supset \dots\supset A^n\supset \dots, $$ where $A^i$ is obtained by evaluating this formal group law on $p^i\bZ_p$. This filtration is subject to the following conditions (see \cite{serre2009lie}):
\begin{enumerate}
	\item for $i\geq 1$, $A^0/A^i$ is finite;
	\item $\bigcap_{i\geq 0} A^i = \{0\}$;
	\item for $i\geq 1$, $A^i/A^{i+1}\isom (\bF_p, +)$;
	\item $A^1[\tors]=A^1[p^{\infty}]$.
\end{enumerate}
By Theorem \ref{torsion-alg-closed}, $A^0[p]$ is finite. Then condition (2) implies that $A^i[p]=0$ for some $i$ and hence $A^i[p^{\infty}]=0$. By condition (4), $A^i[\tors]=0$. Then $A^0[\tors]\hookrightarrow A^0/A^i$ and by condition (1) we find that $A^0[\tors]$ is finite. As its torsion subgroup is finite, the group $A^0$ is torsion-split, i.e. $A^0\isom A^0[\tors]\times A^0/A^0[\tors]$ (see \cite{baer}). This splitting can be shown to be continuous with respect to the induced $p$-adic topologies, so it is in fact an isomorphism of topological groups. $A^0/A^0[\tors]$ is a commutative, torsion-free, pro-$p$ group, so by \cite{clark2019there}[Lemma 20], it is isomorphic to $\prod_{i\in I} \bZ_p$ (as a topological group) for some indexing set $I$. As $A^0$ contains an open subgroup isomorphic to $\bZ_p^g$ \cite{serre2009lie}, we conclude that in fact $|I|=g$.
\end{proof}

It is important to note that, while the torsion subgroup of an abelian variety over $\bQ_p$ is always finite, there is no \emph{uniform} bound on its size among all abelian varieties of a fixed dimension. For example, on the Tate curve $\bQ_p^*/\langle p^n\rangle$, the point $p$ has order $n$.

\section{Abelian varieties over $p$-adically closed fields}
We can use our transfer principle to obtain an analogous structure for abelian varieties over arbitrary $p$-adically closed fields, but the method currently relies on a conjecture which remains unproven:

\begin{conjecture}
\label{conjecture}
Let $\mathcal{A}\ra S$ be an abelian scheme with $S$ quasi-projective over $\Spec(\bZ)$. For any compact subset $C\subset S(\bQ_p)$ (in the induced p-adic topology), there is an integer $N$ so that $|\mathcal{A}_c(\bQ_p)[\tors]|\leq N$ for every fiber $\mathcal{A}_c/\bQ_p$ over $C$.
\end{conjecture}

We can at least prove the conjecture in the case where $\mathcal{A}\ra S$ is a family of elliptic curves:
\begin{proposition}
Conjecture \ref{conjecture} holds when $\mathcal{A}\ra S$ has relative dimension 1, i.e. $\mathcal{A}\ra S$ is a family of elliptic curves.
\end{proposition}
\begin{proof}
It suffices to consider the case where $S$ is integral with generic point $\xi$. If we pullback along the generic point, we get an elliptic curve $\mathcal{A}_K/K$ where $K=K(\xi)$ is the function field of $S$. As the $j$-invariant is a rational function on $\mathcal{A}_K$, it extends to a non-empty Zariski open subset $U\subset S$. Now on the compact subset $C\cap U(\bQ_p)$, the valuation of $j$ is bounded. \cite{silverman2009arithmetic}[VII.6.1] tells us that for an elliptic curve $E/\bQ_p$, the number of components of the special fiber of the N\'eron model is at most $\max\{-v(j(E)), 4\}$, so we conclude that each fiber over $C\cap U(\bQ_p)$ has a uniformly bounded number of sides to its N\'eron polygon, and therefore a uniformly bounded number of torsion points. As $C=U(\bQ_p)\union (S\setminus U)(\bQ_p)$, the claim is proven by Noetherian induction.
\end{proof}

\begin{theorem} Let $K$ be a p-adically closed field and $A/K$ an abelian variety. Then there is an isomorphism of groups $$A(K)\cong V\oplus T,$$ where $V$ is a torsion-free group of order $|K|$ that is uniquely $q$-divisible for every prime $q$ not equal to $p$ and $T$ is a torsion group of finite order. In particular, $V\otimes \bZ[1/p]$ is a $\bQ$-vector space of dimension equal to $|K|$.
\begin{proof}
First, we prove that the torsion subgroup $T$ of $A(K)$ is finite. Arguing as above, we find that there is an abelian scheme $\mathcal{A}\ra S$ with $S$ quasi-projective over $\Spec(\bZ)$ and a $K$-point $s:\Spec(K)\ra S$ so that $A\cong \mathcal{A}_s$ as an abelian variety over $K$. Without loss of generality, we assume $S=\Spec(\bZ[x_1,\dots, x_k]/I)$ is affine. For any $k$-tuple $\alpha = (n_1,\dots, n_k)\in\bZ^k$, consider the $p$-adically compact subset $C_{\alpha}= \{(a_1,\dots, a_k)\in S(\bQ_p): v(a_i)\geq n_i\text{ for } i=1,\dots, k\}$. By Conjecture \ref{conjecture}, there is an integer $N$ so that for each $n$, the following first-order sentence holds for the fiber over each $\bQ_p$-point of $S$ lying in the definable set $C_{\alpha}$: $\forall x, nx=0\implies Nx=0$. By \ref{transfer-theorem}, these sentences holds for our given abelian variety $A/K$. This sequence of sentence is equivalent to the claim that $T$ has exponent dividing $N$, and as $T\subset (\bQ/\bZ)^{2g}$ by \ref{torsion-alg-closed} (where $g=\dim A$), we find that $T$ is finite.

Second, as $T$ is finite, \cite{baer} implies that $A(K)$ is torsion-split, in the sense that $A(K)\cong V\oplus T$ where $V$ is torsion-free.

Third, we prove that $V$ is uniquely $q$-divisible for $q\neq p$. 
Let $N':= N/p^{v_p(N)}$ and $q$ a prime different from $p$. First, we prove that $N'A(K)$ is uniquely $q$-divisible for each $q\neq p$. This is a first-order property which can be verified for the fiber over each $\bQ_p$-point of the definable set $C_{\alpha}$. Indeed, if $B/\bQ_p$ is one such fiber, then writing $B(\bQ_p)\cong \bZ_p^g \oplus B(\bQ_p)[\tors '] \oplus B(\bQ_p)[p^{\infty}]$ (i.e. separating out the $p$-primary part), we can write $P\in B(\bQ_p)$ as $P=(x,y,z)$. Then $N'P=(N'x,0,N'z)$. $\bZ_p^g$ and $B(\bQ_p)[p^{\infty}]$ are both uniquely $q$-divisible, so $N'B(\bQ_p)$ is indeed uniquely $q$-divisible. By the transfer principle, it also holds for $A(K)$. Now let $P\in V$. Then there is a unique $Q\in A(K)$ so that $N'P=N'qQ$ (if an abelian group is $q$-divisible for $q\neq p$, then it is $m$ divisible for $p\nmid m$). Writing $Q=(v,t)$, we have $N'P=N'qv\in V$. As $V$ is torsion-free, we must have $P=qv$. That $V$ is uniquely divisible follows from the uniqueness of $Q$. 

Finally, it remains to prove the cardinality statements. In \cite{pop2010henselian}, it is proved that for any ample field $F$ (in the sense of Definition \ref{ample})and smooth curve $C/F$ with $C(F)\neq\emptyset$, $|C(F)|=|F|$. As p-adically closed fields are ample, we conclude that $|A(K)|=|K|$ (simply choose any smooth curve on $A$ that passes through the identity element). Since $T$ is finite and $K$ is not, we must have $|V|=|K|$. That $V\otimes \bZ[1/p]$ is a $\bQ$-vector space follows immediately from the above; that its dimension as a $\bQ$-vector space is equal to $|K|$ follows from the exact same argument as for the case of real closed fields.
\end{proof}
\end{theorem}
\end{document}

