\externaldocument{intro}

\chapter{Abelian varieties over real closed fields}
The first application of our transfer principle will be to the case of abelian varieties over real closed fields.

\begin{definition}
A field $R$ is said to be \emph{real closed} if it satisfies any of the following equivalent conditions:
	\begin{itemize}
		\item $R$ is formally real (i.e. $-1$ is not a sum of squares in $R$) and it has no proper formally real algebraic extension.
		\item There is a total order on $R$ making it an ordered field such that every positive element has a square root in $R$ and every polynomial of odd degree has a root in $R$.
		\item $R$ is not algebraically closed, but $R(\sqrt{-1})$ is algebraically closed.
		\item $R$ is elementarily equivalent to $\bR$, in the sense that if $\phi$ is any sentence in the language of fields, then $R\models \phi$ if and only if $\bR\models \phi$.
	\end{itemize}
\end{definition}

Naturally, it is the fourth condition which is of interest to us. In the next section, we reproduce the relevant results for abelian varieties over $\bR$, as given in \cite{gross-harris}; we use our transfer principle to obtain analogous results for abelian varieties over arbitrary real closed fields in the section thereafter.

\section{Abelian varieties over $\bR$}
The result of this section can be found in \cite{gross-harris}. The proof is real analytic in nature and is reproduced here to illustrate that Weil's philosophy of the Lefschetz principle holds equally well in the setting of real closed fields.

\begin{theorem}
\label{structure-of-real-AVs}
Let $A/\bR$ be an abelian variety of dimension $g$ defined over $\bR$. Then there is an integer $d$ with $0\leq d\leq g$ so that $$A(\bR)\isom (\bR/\bZ)^g \times (\bZ/2\bZ)^d$$
\end{theorem}
\begin{proof}
Let $A(\bR)^0$ be the connected component of the identity in $A(\bR)$. Then since $A(\bR)^0$ is a compact, connected, abelian, real Lie group of dimension $g$, it is a torus: $A(\bR)^0\isom (\bR/\bZ)^g$ (see \cite{compactliegroups}, for instance). Since this group is divisible, the exact sequence $$0\ra A(\bR)^0 \ra A(\bR)\ra A(\bR)/A(\bR)^0\ra 0$$ splits, and we find that $A(\bR)\isom (\bR/\bZ)^g\times A(\bR)/A(\bR)^0$. Now it suffices to demonstrate that the component group $A(\bR)/A(\bR)^0$ is a 2-group whose rank falls in the specified range.

Consider the norm map $N:A(\bC)\ra A(\bR)$ given by $N(P)=P+\bar{P}$. $N$ is a continuous homomorphism and $A(\bC)$ is compact and connected, so the image $N(A(\bC))$ is a closed, connected subgroup of $A(\bR)$ containing the identity. On the other hand, the image $N(A(\bC))$ contains the finite index subgroup $2A(\bR)$ and so is itself finite index and therefore open (NOTE: perhaps say why $2A(\bR)$ is finite index? it follows from an easy galois cohomology argument, but it is also probably easier than that? finite index + closed implies open is standard top. group fact). We conclude that $N(A(\bC))=A(\bR)^0$ and furthermore that $A(\bR)/A(\bR)^0$ is killed by 2. $A(\bR)$ is compact, so the component group is finite. Thus $A(\bR)\isom (\bR/\bZ)^g\times (\bZ/2\bZ)^d$ for some $d$. The bound on $d$ follows from Proposition \ref{torsion-C}
\end{proof}

\section{Abelian varieties over real closed fields}
Much of the content in this section can be found in the author's previous work \cite{lowry2023abelian}, but is here reformulated using the machinery developed in Section 1.5.

As mentioned above (Theorem \ref{frey-jarden}), Frey and Jarden show in \cite{frey-jarden} that for every abelian variety of dimension $g>0$ defined over an algebraically closed field $K$ of characteristic zero, there is an isomorphism of groups $$A(K)\cong V\times(\bQ/\bZ)^{2g},$$ where $V$ is a $\bQ$-vector space of dimension equal to $|K|$, the cardinality of $K$. We prove an analogous result for real closed fields:

\begin{theorem}
\label{structure-of-realclosed-AVs}
Let $R$ be a real closed field and $A/R$ an abelian variety of dimension $g>0$. Then there is an isomorphism of groups $$A(R)\cong V \times (\bQ/\bZ)^g\times (\bZ/2\bZ)^d$$ for some $0\leq g \leq d$, where $V$ is a $\bQ$-vector space of dimension $|R|$, the cardinality of $R$.
\end{theorem}

We first prove that $A(R)$ has the desired rank $\dim_{\bQ}A\otimes_{\bZ} \bQ$.

\begin{proposition}
\label{prop-freepart}
Let $A/R$ be an abelian variety of dimension $g>0$ over a real closed field. Then the rank of $A(R)$ is equal to the cardinality of $R$.
\end{proposition}
\begin{proof}
It is known that real closed fields are ``ample" (or ``large" as in Pop \cite{pop}) in the sense of Definition \ref{ample} above (NOTE: this seems to be well-known, but I can't find a specific reference for a proof; if you wanted to prove it, it would be (essentially) exactly the same argument as this very proposition!). Thus Theorem \ref{fehm-petersen} demonstrates that $A(R)$ has infinite rank. If $R$ countable, then so too is $A(R)$ and the result follows.\newline \indent If $R$ is uncountable, then one may employ the implicit function theorem for real closed fields (as in \cite{real}, Corollary 2.9.8) to find at least $|R|$ points in $A(R)$. Indeed, consider an affine open $U=\Spec(R[x_1,\dots, x_n]/(f_1,\dots, f_m))$ around the identity $e\in A(R)$. As $A$ is smooth, it is a local complete intersection, so we may assume that $n-m=g$. Consider the map $R^n\ra R^m$ given by the $f_i$. Again since $A$ is smooth, the Jacobian matrix $(\partial f_i / \partial x_j)$ has maximal rank $m$ at $e$ and thus there is some $m\times m$ minor of this matrix which is non-vanishing. After relabeling coordinates we may assume that $e$ belongs to the open set where $\det( (\partial f_i / \partial x_j)_{g+1\leq j\leq n})\neq 0$. Now the implicit function theorem guarantees the existence of a non-empty open subset $C\subset R^g$ and a definable function $h:C\ra R^m$ so that $(a_1,\dots, a_g, h(a_1,\dots, a_g))\in A(R)$ for every $(a_1,\dots, a_g)\in C$. As non-empty open subsets in $R^g$ have cardinality $|R|$, this demonstrates at least $|R|$ points of $A(R)$. As the cardinality of $A(R)$ is certainly at most that of $R$, we have $|A(R)|=|R|$. \newline \indent The torsion subgroup $T\subset A(R)$ is at most countable, so $|A(R)/T|=|R|$ since we assume $R$ to be uncountable. Finally, $A(R)/T$ is isomorphic to a subgroup of $A(R)\otimes \bQ$, so $|A(R)\otimes \bQ|=|R|$ as well. The result follows from a simple fact of cardinal arithmetic: if $V$ is a $\bQ$-vector space of uncountable cardinality $\kappa$, then $\dim_{\bQ} V=\kappa$.
\end{proof}

\begin{proposition}
\label{torsion-prop-real-closed}
Let $R$ be a real closed field, $A/R$ an abelian variety of dimension $g>0$. Then $A(R)[\tors]\isom (\bQ/\bZ)^g\times (\bZ/2\bZ)^d$ for some $0\leq d\leq g$.
\end{proposition}
\begin{proof}
Arguing exactly as in the proof of Proposition \ref{torsion-alg-closed-transfer}, there is an abelian scheme $\mathcal{A}/S$ with $S$ affine and finite type over $\Spec(\bZ)$ together with an $R$-point $s:\Spec(R)\ra S$ such that the fiber $\mathcal{A}_s/R$ is isomorphic to $A/R$ as an abelian variety over $R$. Now for fixed odd $p$ and fixed natural numbers $n$ and $g$, the statement ``there are exactly $p^{ng}$ elements killed by $p^n$" is a first-order statement in the language of groups which, by Theorem \ref{structure-of-real-AVs}, is known to hold for the fiber over each $\bR$-point of $S$. By Theorem \ref{transfer-theorem}, it must hold for $A$ as well. 

Similarly, the sentence ``for some $0\leq d\leq g$, there are exactly $2^{g+d}$ elements killed by 2" is a first-order sentence in the language of groups which, again by Theorem \ref{structure-of-real-AVs}, is known to hold for the fiber over each $\bR$-point of $S$, and so by Theorem \ref{transfer-theorem}, must hold for $A$ as well. Now, for each $n$, the first-order sentence ``for every $0\leq g\leq d$, if there are exactly $2^{ng+d}$ elements killed by $2^n$, then there are exactly $2^{(n+1)g+d}$ elements killed by $2^{n+1}$" is known to hold for the fiber over each $\bR$-point of $S$ and so again by Theorem \ref{transfer-theorem}, these statements holds for $A$. This sequence of statements amounts to the claim that there is some \emph{fixed} $0\leq d \leq n$ so that for each $n$, $|A[2^n|(R)|=2^{ng+d}$. 

As in the case of an algebraically closed field, purely group theoretic considerations now imply that the torsion subgroup of $A$ has the desired form.
\end{proof}

NOTE: I think we came up with a proof of this fact that doesn't require the model theory, at least when $A$ is principally polarized. It was something like ``look at the action of conjugation on the $\bZ/p\bZ$-vector space $A[p](R(\sqrt{-1}))$ and use the Weil pairing." Should we include this?

We are now ready to prove Theorem \ref{structure-of-realclosed-AVs}:

\begin{proof}
The above proposition shows that the torsion subgroup $T$ of $A$ is a divisible group times a finite group; it follows from \cite[Theorem 8.1]{baer} that $A(R)$ is torsion-split, i.e. $A(R)\cong V\oplus T$, where $V$ is torsion free. We need only to verify that $V$ is uniquely divisible. First note that $2A(R)$ is uniquely divisible. Indeed, it is uniquely $n$-divisible for each $n$, as this is a first-order statement in the language of groups which can be verified over the $\bR$-points in a definable family containing $A$. Now given any $v\in V$, $2v\in 2A(R)$ whence $2v/2n\in A(R)$, i.e. there is some $y\in A(R)$ so that $2v=2ny$. Writing $y=w+t$ for $w\in V$ and $t\in T$, we find that $2v=2nw$, and since $V$ is torsion-free we must have $v=nw$. That $V$ is \emph{uniquely} divisible follows since $2A(R)$ is uniquely divisible. Finally, that $V$ has the given dimension follows from Proposition \ref{prop-freepart}.
\end{proof}

\begin{corollary}
Let $R$ be a real closed field, $p$ an odd prime, and $A/R$ an abelian variety of dimension $g$. Then the Tate module $T_p(A)$ is a free $\bZ_p[G]$-module of rank $g$ where $G=\Gal(R(\sqrt{-1})/R)$.
\end{corollary}
\begin{proof}
This follows immediately from the description of $A[p^n](R)$ above.
\end{proof}

The final result of this section examines how the isomorphism type of the group of rational points of a real abelian variety varies in a family. The structure theorem proved above demonstrates that this isomorphism type is entirely determined by the number of 2-torsion points, giving evidence to suggest that this isomorphism type should be locally constant. As an illustrative example, consider the family $\mathcal{E}\ra \Spec(\bZ[a,b][1/(4a^3+27b^2)])=S$, where $\mathcal{E}$ is the closed subscheme of $\bP^2\times S$ cut out by $Y^2Z=X^3+aXZ^2+bZ^3$. One notices that the elliptic curves with \emph{connected} real locus (hence, exactly two 2-torsion points) occur over the fibers with $4A^3+27B^2>0$, and that those with \emph{disconnected} real locus (hence, exactly four 2-torsion points) occur over the fibers with $4A^3+27B^2<0$. We show that this is a more general phenomenon.

\begin{proposition}
\label{deformation}
Let $\mathcal{A}\ra S$ be an abelian scheme of dimension $g$ with $S$ of finite type over a real closed field $R$. Then the function $S(R)\ra \bN$ sending $s\in S(R)$ to $|\mathcal{A}_s[2](R)|$ is locally constant with respect to the order topology on $S(R)$ inherited from the ordering on $R$. In particular, the isomorphism class of the fiber (as an abelian group) is constant on the connected components of $S(R)$ (in the order topology).
\end{proposition}
\begin{proof}
We seek to apply a suitable version of the following elementary fact from the theory of real manifolds, whose proof (as in \cite{milnor}) is a simple application of the inverse function theorem: let $f:M\ra N$ be a smooth map between real manifolds of the same dimension with $M$ compact. If $U$ is the (open) subset of $N$ consisting of regular values, then the function $U\rightarrow \bN$ which sends $y\in U$ to $|f\inv(y)|$ is locally constant. 

Now, the 2-torsion subgroup scheme $\mathcal{A}[2]$ is \'etale over $S$, as it is the pullback of the \'etale morphism $[2]:\mathcal{A}\ra A$. For each $s\in S(R)$, the set $\mathcal{A}_s[2](R)$ is finite and non-empty by Theorem \ref{structure-of-realclosed-AVs}, so at the level of $R$-points, we have a surjective \'etale morphism $\mathcal{A}[2](R)\ra S(R)$ (where an \'etale morphism of algebraic sets is a regular morphism which induces an isomorphism of tangent spaces). A definable version of the inverse function theorem is used in \cite[Proposition 8.1.2]{real} to conclude that $\mathcal{A}[2](R)\ra S(R)$ is a local diffeomorphism. The first result now follows from the exact same argument as in the case of real manifolds. The isomorphism class of a real abelian variety (as an abelian group) is completely determined by its 2-torsion subgroup, so the second statement follows.
\end{proof}