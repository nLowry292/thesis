\documentclass[showabstract,showacknowledgments,showpreface,showdedication,12pt]{iuphd} 


%%% -- Packages -- %%%
\usepackage[utf8]{inputenc}
% \usepackage[english]{babel}
% \usepackage[margin = 1.25in]{geometry} %1.5in
\usepackage{amsthm}
\usepackage{amsmath}
\usepackage{xr}
\usepackage{amssymb}
\usepackage[scr]{rsfso}
% \usepackage{csquotes}
\usepackage{bm,bbm}
% \usepackage{color}
% \usepackage[dvipsnames]{xcolor}
% \usepackage{graphicx}
% \usepackage{caption, subcaption}
% \usepackage{array}
\usepackage{tikz}
\usetikzlibrary{matrix,arrows}
\usepackage{tikz-cd}
\usepackage[framemethod=TikZ]{mdframed}
% \usepackage{youngtab}
\usepackage{mathtools}
% \usepackage[normalem]{ulem}
% \usepackage[all]{xy}
% \usepackage{rotating}
% \usepackage{extarrows}
% \usepackage{accents}
% \usepackage{leftidx}
% \usepackage{upgreek}
% \usepackage{xr-hyper}
% \usepackage[draft]{todonotes}
% \usepackage{proofread}
\usepackage{hyperref}
% \usepackage{qtree}
\usepackage{tocloft}
%\usepackage{pdfpages} 
\usepackage{macros}
% For title and abstract page

\title{Abelian Varieties over large non-algebraically closed fields}
\author{Nathanial D. Lowry}
\date{DATE TBD} % Completion date of Dissertation
\department{Mathematics} % Change this to your department if not Mathematics

% For acceptance and abstract page

\committeechair{Michael Larsen, PhD}
\readertwo{Valery Lunts}
\readerthree{Matthias Strauch}
\readerfour{Larry Moss} 
\defensedate{TBD} % Date of PhD defense

% For Copyright Page
%\cryear{Year} % Copyright year

\renewcommand{\contentsname}{
% Change font size to whatever you want.
\large Table of Contents % Change TOC title to whatever you want.
}

\begin{document}
\maketitle
\acceptancepage


% This page is optional
%\copyrightpage


% This page is optional but generally included

\begin{acknowledgements}
I find myself in the enviable position of being surrounded by a wonderfully supportive and loving group of friends and family. I would like to use this space to express my gratitude to a few of these people in particular:

Michael Larsen, my advisor, for his support on this thesis, and for allowing me to draw from his well of patience and kindness which never runs dry;

R. Alexander Glickfield, whose friendship and camaraderie are a beacon of light in the otherwise dark and gloomy skies of graduate school;

Zachary Finn, for teaching me to have the courage of my convictions;

Dylan Spence and A. Thomas Yerger, for their many stimulating conversations, mathematical and otherwise;

David and Jamie Lowry, my parents, for their unending love and their continued encouragement in my academic career;

Roberta Glotzbach, my grandmother, for whom the words to express my gratitude would overflow the bounds of this page;

and finally, Lauren Lykins and Theodore Lowry, my wife and my son, for whom there simply are no words to express the extent of my love.
\end{acknowledgements}

% This page is optional

% \begin{dedication}
% This is the (optional) dedication page. Per Graduate School standards, this page should appear with no title and should be centered horizontally and vertically.
% \end{dedication}

% % This page is optional

% \begin{preface}
% This is the (optional) preface page which can be used if you wish. This page should appear after the dedication (or acknowledgements page if there is no dedication page) and before the abstract page.
% \end{preface}

% % This page is required

\begin{abstract}
      In this thesis, we determine, to as great an extent as possible, the group structure of the $K$-valued points of an abelian variety defined over $K$ when $K$ is either a real closed field or a $p$-adically closed field. Using the known structure of abelian varieties over $\bR$ and $\bQ_p$, we employ a model-theoretic transfer principle to pass this information to abelian varieties over $K$.\hspace{2cm}
\end{abstract}

\newpage

% This page is required

\tableofcontents

\newpage

\pagenumbering{arabic}
\bibliographystyle{alpha}
\bibliography{references}
\chapter{Introduction}

Abelian varieties play an important role for the algebraic and arithmetic geometer alike. Their rigid geometric structure simplifies the analytic questions one might ask, while their arithmetic structure provides a fruitful link between geometry and number theory. Elliptic curves are the first and simplest examples of abelian varieties. The analytic theory of elliptic curves traces back to unpublished works of Gauss, but the first systematic studies are attributed to Abel and Jacobi, who, in giving a basis for the theory of elliptic functions, discovered higher-dimensional abelian varieties known today as Jacobians. Some 30 years later, Weierstrass discovered the $\wp$-function - an explicit, albeit highly transcendental, parametrization of elliptic curves. It is also around this time that Riemann introduced the idea of manifolds, giving a firm foundation to the theory of higher-dimensional abelian varieties; he introduced the Riemann bilinear relations as well as theta functions into the theory of abelian varieties. In 1869, Weierstrass used Riemann's foundational work to prove that abelian functions satisfy certain additional algebraic relations, attracting attention from the algebraists. Kronecker used this algebraic structure of elliptic curves to solve Hilbert's 12th problem for the case of imaginary quadratic fields and in 1922, Mordell proved that the group of $\bQ$-points on an elliptic curve is in fact finitely generated. In the 1928, Weil greatly extended Mordell's and in the 1940s, he gave the theory of abelian varieties its modern formulation. While Grothendieck's introduction of schemes in the early 1960s largely supplanted Weil's theory,  Weil's treatises remain highly relevant and highly readable to this day.

In this thesis, we use real and $p$-adic analytic techniques to reveal the algebraic structure of abelian varieties over $\bR$ and $\bQ_p$. We employ a model-theoretic transfer principle to deduce arithmetic properties of abelian varieties over real closed and $p$-adically closed fields.

\section{Abelian Varieties}
\begin{definition}
 Let $k$ be a field. An abelian variety over $k$ is a proper and connected group variety. 
\end{definition}

\begin{remark}
For the remainder of this thesis, all abelian varieties are assumed to be positive-dimensional.
\end{remark}

While it may not seem so obvious from the definition, the conditions required of an abelian variety are quite strong. In particular, they will imply that the group law is always commutative and that abelian varieties are projective.

\begin{definition}
A homomorphism $f:A\ra B$ of abelian varieties is called an \emph{isogeny} if $f$ is surjective with finite kernel. The degree of an isogeny is the order of the kernel (as a group scheme).
\end{definition}

As abelian varieties are commutative group schemes, they come with a multiplication-by-$n$ map $[n]:A\ra A$ for each $n\in\bZ$. We denote the kernel of this map by $A[n]$ and we frequently identify $A[n]$ with its $\bar{k}$-points. The most important basic result for us is the following, whose proof we delay until Proposition \ref{torsion-alg-closed-lefschetz}:

\begin{proposition}
\label{torsion-alg-closed}
The degree of $[n]:A\ra A$ is $n^{2g}$, where $g=\dim A$. In particular, if $k$ is of characteristic zero, then $A[n](\bar{k})\isom (\bZ/n\bZ)^{2g}$ and $A(\bar{k})[\tors]\cong (\bQ/\bZ)^{2g}$.
\end{proposition}
%\begin{proof}
%We first recall two facts. First, if $X$ and $Y$ are proper varieties of the same dimension and $f:X\ra Y$ is surjective, then $f^*$ induces a finite extension $k(X)/k(Y)$ of degree $d$. Now if $D_1,\dots, D_n$ are Cartier divisors on $Y$, we find the following relation between intersection numbers: $$(f^*D_1,\dots, f^*D_n)_X=d(D_1,\dots, D_n)_Y.$$ Second, if $X$ is an abelian variety and $\mathcal{L}$ is a line bundle on $X$, then $[n]^*\mathcal{L}\cong \mathcal{L}^{\frac{n^2+n}{2}}\otimes [-1]^*\mathcal{L}^{\frac{n^2-n}{2}}$. This follows from the so-called ``theorem of the cube". 
%
%Now let $D$ be an ample, symmetric divisor on $A$; such a divisor exists because there is an ample line bundle $\mathcal{L}$ over $A$, since $A$ is projective, and $\mathcal{L}\otimes [-1]^*\mathcal{L}$ will be ample and symmetric). By the second fact, we have that $[n]^*D\sim n^2 D$ and by the first fact we find that $\text{deg}([n])=n^{2g}$ (intersect $[n]^*D$ with itself $g$-times). That $A(\bar{k})[\tors]$ is of the given form is an easy exercise in group theory.
%\end{proof}

If $A$ is an abelian variety defined over $k$, then the multiplication map and therefore the multiplication-by-$n$ maps are all defined over $k$. Thus, the Galois group $G:=\Gal(\bar{k}/k)$ acts on each $A[n]$, and we get a Galois representation $G\ra GL_{2g}(\bZ/n\bZ)$. Representation theory is significantly easier in characteristic zero, so we are lead to consider instead the \emph{Tate module} $T_l(A):=\varprojlim A[l^n]\cong \bZ_l^{2g}$ with transition maps given by $[l]:A[l^{n+1}]\ra A[l^n]$, where $l$ is a prime different from the characteristic of $k$. Since the multiplication-by-$l$ map is Galois-equivariant, $T_l(A)$ comes equipped with a natural Galois action and we get a representation $G\ra GL_{2g}(\bZ_l)$.  \emph{Many} interesting properties of abelian varieties can be extracted by a careful analysis of this Galois representation. As just one example, consider the celebrated theorem of Faltings: two elliptic curves $E_1$ and $E_2$ over a number field $K$ are $K$-isogenous if and only if their $l$-adic Tate modules $T_l(E_1)$ and $T_l(E_2)$ are isomorphic as $\Gal(\bar{K}/K)$-modules.

Associated to an abelian variety $A/k$ is a \emph{dual abelian variety} $A^{\vee}/k$, sometimes denoted $\Pic^0(A)/k$: it is the identity component of the group scheme representing moduli of line bundles on $A$. As such, the dual abelian variety comes with a line bundle $\mathcal{P}$ on $A\times A^{\vee}$ called the \emph{Poincar\'e bundle}. As one might expect given the name of \emph{dual} abelian variety, there is a natural isomorphism of abelian varieties $\iota_A:A\simeq A^{\vee\vee}$; moreover, associated to each homomorphism (resp. isogeny) $f:A\ra B$ of abelian varieties, there is a corresponding dual homomorphism (resp. dual isogeny) $f^{\vee}:B^{\vee}\ra A^{\vee}$.

\begin{definition}
A polarization of an abelian variety $A/k$ is an isogeny $\lambda:A\ra A^{\vee}$ satisfying the following two conditions:
\begin{enumerate}
	\item $\lambda$ is \emph{symmetric} in the sense that $\lambda^{\vee}\circ \iota_A = \lambda$
	\item The line bundle $$\Delta^*(\id\times \lambda)^*\mathcal{P}=(\id,\lambda)^*\mathcal{P},$$
	on $A$ is ample, where $\Delta:A\ra A\times A$ is the diagonal morphism and $\mathcal{P}\in\Pic(A\times A^{\vee})$ is the Poincar\'e bundle.
\end{enumerate} 
The \emph{degree} of a polarization is its degree as an isogeny. A \emph{polarized abelian variety} is a pair $(A,\lambda)$, where $\lambda$ is a polarization of the abelian variety $A$. If $\lambda$ has degree 1, then $(A,\lambda)$ is said to be \emph{principally polarized}.
\end{definition}

It is known that every abelian variety $A/k$ admits a polarization over $k$. This will be important for us, as one can show that \emph{polarized} abelian varieties $(A,\lambda)$ have finite automorphism groups, making them particularly well-suited to the formation of well-behaved moduli problems (see \cite{mumfordAV}[Theorem 5, pg. 207]).

As abelian varieties are projective, one such moduli problem that will be of particular importance for us is the classification of closed subschemes of projective space. Consider the functor $\Sch^{op}\ra \Set$ from the category of locally noetherian schemes to sets given by $$\Hilbit(n)(T)=\{Y\subset \mathbb{P}^n_T: Y\text { is closed and flat over } T\}.$$ In \cite{FGAHilbert}, Grothendieck proved that this functor is representable by a scheme $\Hilb(n)$ in the sense that the set of $T$-valued points $\Hom(T,\Hilb(n))$ is naturally identified with $\Hilbit(n)(T)$. Moreover, he proved that $\Hilb(n)$ naturally decomposes into a disjoint union of schemes $\Hilb(n,\Phi)$, where $\Phi$ ranges over the set of rational polynomials. The scheme $\Hilb(n,\Phi)$ represents the functor that classifies closed subschemes $Y$ of $\bP^n_T$, flat over $T$, with Hilbert polynomial $\Phi$. Furthermore, he proved that each $\Hilb(n,\Phi)$ is in fact projective over $\Spec(\bZ)$. The main technical result of this thesis, Lemma \ref{comprehensive-family}, amounts to excising precisely the abelian varieties from the family of closed subschemes parametrized by $\Hilb(n,\Phi)$.

Crucially, one can also discuss abelian \emph{schemes}, which are families of abelian varieties defined over some base scheme of parameters. 

\begin{definition}
A abelian scheme $\mathcal{A}/S$ (of relative dimension $g$) is a proper, smooth, group scheme over $S$ whose geometric fibers are are connected (and of dimension $g$).
\end{definition}

These conditions of course imply that the fiber of $S$ over any field-valued point is an abelian variety (of dimension $g$).

(NOTE: the template for this document I was given uses this fancy "example" set-up. Should I change it back to the default?)
\begin{example}
An important example to which we frequently return is the family of elliptic curves $\mathcal{E}/S$ over the base $S=\Spec(\bZ[a,b][1/(4a^3+27b^2)])$ and where $\mathcal{E}$ is the closed subscheme of $\bP^2\times S$ cut out by $Y^2Z=X^3+aXZ^2+bZ^3$. If $k$ is a field, then  giving a $k$-point $s:\Spec(k)\ra S$ is equivalent to specifying two points $a,b\in k$ satisfying $4a^3+27b^2\neq 0$. The fiber $\mathcal{E}_s/k$ is (isomorphic to) the elliptic curve $E/k$ given in Weierstrass form by $y^2=x^3+ax+b$. This family of elliptic curves is \emph{comprehensive} in the sense that it contains \emph{all} elliptic curves defined over fields of characteristic zero: for every field $k$ of characteristic zero and every elliptic curve $E/k$, there is a $k$-point $s:\Spec(k)\ra S$ such that $E/k$ is isomorphic (as an elliptic curve over $k$) to $\mathcal{E}_s/k$.
\end{example}

\section{Abelian Varieties over $\bC$}

Abelian varieties were first discovered by means of their intimate connection with abelian functions, which are meromorphic functions satisfying a certain number of periodicity conditions. One can learn a great deal of information about abelian functions by studying differential forms on complex algebraic curves. Associated to any curve $C$ of genus $g$ is an abelian variety $J$ of dimension $g$ called the \emph{Jacobian} of $C$. The rigid structure of the Jacobian variety greatly simplifies the study of differential forms on $C$.

The original impetus for studying elliptic curves came from a detailed analysis of so-called ``elliptic functions" --- meromorphic functions $f$ that are ``doubly periodic" in the sense that there are real-linearly independent complex numbers $\omega_1$ and $\omega_2$ so that $f(z)=f(z+a\omega_1+b\omega_2)$ for all $z\in \bC$ and $a,b\in\bZ$. Some elementary complex analysis will show us that any such non-constant $f$ must have at least 2 poles in the fundamental domain $D=\{a\omega_1+b\omega_2: 0\leq a,b< 1\}$: that $f$ is holomorphic is ruled out by Liouville's Theorem, as $f$ is completely determined by its valued on the compact set $\bar{D}$; that $f$ has a single simple pole is ruled out by a simple application of the residue theorem, where the periodicity of $f$ guarantees that the sum of the residues in the fundamental domain must be 0. With this in mind, it is natural to consider the function $$\sum\limits_{\lambda\in \Lambda} \frac{1}{(z-\lambda)^2}.$$ Alas! This sum does not converge. With a small modification, we arrive at the Weierstrass $\wp$-function: $$\wp(z;\Lambda) = \frac{1}{z^2} + \sum\limits_{\lambda\in\Lambda\setminus\{0\}} \frac{1}{(z-\lambda)^2}-\frac{1}{\lambda^2}.$$ 

As a historical aside, the fancy script letter $\wp$ was chosen to designate this function by its inventor Weierstrass and has been retained by posterity. This symbol is used almost exclusively in reference to this function, although it is used somewhat perversely on Wikipedia to refer to the powerset.

One can check manually that both $\wp$ and its derivative $\wp'$ are doubly periodic with respect to the lattice $\Lambda$. In fact, one can even prove that $\bC(\Lambda)=\bC(\wp,\wp')$, i.e. \emph{every} elliptic function with respect to $\Lambda$ is a rational function in $\wp$ and $\wp'$. Moreover, one can prove that $\wp$ satisfies the differential equation $(y')^2 = 4y^3-g_2y-g_3$, where $g_2$ and $g_3$ are constants depending on $\Lambda$. In other words, the $\wp$-function and its derivative $\wp'$ parametrizes an elliptic curve! Amazingly enough, \emph{every} elliptic curve can be parametrized in such a way:

\begin{theorem}
\label{uniformization}
Let $E/\bC$ be an elliptic curve. Then there is a lattice $\Lambda\subset \bC$ so that $\phi:\bC/\Lambda\ra E(\bC)$ defined by $\phi(z+\Lambda)=[\wp(z;\Lambda):\wp'(z;\Lambda):1]$ and $\phi(\Lambda) = [0:1:0]$ is an isomorphism of complex analytic Lie groups.
\end{theorem}

One might describe the correspondence between elliptic curves and elliptic functions thus: an elliptic function is a meromorphic function on some complex elliptic curve $\bC/\Lambda$, regarded as a $\Lambda$-periodic meromorphic function on $\bC$.

The obvious next step is to consider the so-called ``abelian functions" --- meromorphic functions of $n$ complex variables with $2n$ real-independent periods. Naturally, it was found that an abelian function is a meromorphic function on some complex abelian variety!

However, as is often the case in mathematics, the story is much more complicated in higher dimensions. As one might expect from the case of elliptic curves, an abelian variety $A/\bC$ is still a complex torus, i.e. there is a lattice $\Lambda\subset \bC^g$ so that $A(\bC)\isom \bC^g/\Lambda$ as complex Lie groups. However, not all complex tori are made equally --- there are non-trivial conditions on the lattice $\Lambda$ that must be satisfied in order for $\bC^g/\Lambda$ to be an abelian variety. Without going into too much detail, one proves that in order for a complex torus to be an abelian variety, it is necessary and sufficient that there be a non-degenerate Hermetian form on $\bC^g$ whose imaginary part is integer-valued on $\Lambda$. Such a form is called a \emph{Riemann form} for $T=\bC^g/\Lambda$. One can use this Riemann form to construct \emph{theta functions}, which play the role of the Weierstrass $\wp$-function. A polarization of an abelian variety $A/\bC$ is the choice of such a Riemann form.

In any case, the description of abelian varieties as complex tori affords us an easy proof of Proposition \ref{torsion-alg-closed} in the case $k=\bC$:
\begin{proposition}
\label{torsion-C}
Let $A/\bC$ be an abelian variety of dimension $g$. Then $A[\tors]\isom (\bQ/\bZ)^{2g}$.
\end{proposition}
\begin{proof}
By the uniformization theorem, $A(\bC)\isom \bC^g/\Lambda$ for some lattice $\Lambda\subset \bC^g$. If $\Lambda = \oplus_{i=1}^{2g}\bZ\cdot \tau_i$, then we see by inspection that the points of order $n$ in $\bC^g/\Lambda$ are (the images of) the points $\sum_{i=1}^{2g}a_i\tau_i/n$ with $0\leq a_i <n$. Thus, there are a total of $n^{2g}$ points of order $n$. That $A[n](\bC)\isom (\bZ/n\bZ)^{2g}$ and that therefore $A[\tors]\isom (\bQ/\bZ)^{2g}$ follow from purely group-theoretic considerations.
\end{proof}

\section{Abelian Varieties over $\bQ$}
The most fundamental result for abelian varieties over $\bQ$ (or more generally over number fields) is the celebrated theorem of Mordell and Weil:
\begin{theorem}
\label{mordell-weil}
Let $K$ be a number field and $A/K$ an abelian variety. Then $A(K)$ is a finitely generated abelian group.
\end{theorem}

Given an abelian variety $A$ defined over a number field $K$, the theorem tells us we have $A(K)\isom \bZ^r \oplus T$, where $T$ is the finite torsion subgroup. The determination of the rank $r$ and the structure of the torsion subgroup $T$ remain active areas of research.

\section{Abelian varieties over other fields}

The theory of abelian varieties of $\bR$ and $\bQ_p$ is quite well-understood, but we will discuss these particular cases in detail below. 

There is an extension of the Mordell-Weil Theorem due to N\'eron:

\begin{theorem}
\label{neron-mordell-weil}
Let $K$ be a field which is finitely generated over its prime field and let $A/K$ be an abelian variety. Then $A(K)$ is a finitely generated abelian group.
\end{theorem}

On the other end of the spectrum, if $k$ is algebraically closed, then $A(k)$ is a divisible group. In particular $A(k)[\tors]$ is divisible; a theorem of Baer \cite{baer} tells us that $A(k)$ is thus torsion split, i.e. $A(k)\isom V\oplus A(k)[\tors]$. $V$ is torsion-free and \emph{uniquely} divisible, so in fact $V$ is a $\bQ$-vector space; the dimension of this $\bQ$-vector space was calculated by Frey and Jarden in \cite{frey-jarden}:

\begin{theorem}
\label{frey-jarden}
Let $K$ be an algebraically closed field which is not the algebraic closure of a finite field and let $A/K$ be an abelian variety. Then the rank of $A(K)$, $\dim_{\bQ} A(K)\otimes \bQ$, is equal to the cardinality of $K$.
\end{theorem}

Note that we exclude the case $K=\bar{F_q}$, as in this case $A(K)$ is a torsion group, so $A(K)\otimes \bQ=0$.

In particular, we come to the exact opposite conclusion of the Mordell-Weil theorem: the rank of $A(k)$ is always infinite when $k$ is algebraically closed (and not equal to $\bar{\bF_q}$). This leads us to the following definition:

\begin{definition}
A field $K$ is called \emph{anti-Mordell-Weil} (or simply \emph{AMW}) if for every abelian variety $A/K$ (remember our convention about abelian varieties being positive-dimensional), the group $A(K)$ has infinite rank.
\end{definition}

We have just seen that algebraically closed fields of characteristic zero are AMW; $\bR$ and $\bQ_p$ will turn out to be AMW. In fact, a field will be AMW if it is ``large" in the following sense:

\begin{definition}
\label{ample}
A field $F$ is called \emph{ample} (or \emph{large}) if for every smooth $F$-curve $C$, either $C(F)=\emptyset$ or $C(F)$ is infinite.
\end{definition}

We have the following recent theorem of Fehm and Petersen:

\begin{theorem}
\label{fehm-petersen}
Let $F$ be an ample field that is not algebraic over a finite field. Then $F$ is AMW.
\end{theorem}

\section{The Lefschetz Principle}
It is quite common to reduce questions about algebraic geometry over an arbitrary algebraically closed field of characteristic zero to questions about algebraic geometry over $\bC$ by appealing to the \emph{Lefschetz princple}, which states, roughly, that algebraically geometry over algebraically closed fields of characteristic zero is ``the same" as algebraic geometry over $\bC$. The idea is that statements about varieties and morphisms are, at bottom, statements that demonstrate solutions to certain systems of polynomial equations; that the varieties and morphisms themselves can be defined over a finitely generated extension of $\bQ$; and that any such field can be embedded into $\bC$. Weil summarizes this point excellently in his foundational work \cite{weil1946foundations}: ``S. Lefschetz has observed on various occasions, whenever a result, involving only a finite number of points and varieties, can be proved in the `classical case' where the universal domain is the field of all complex numbers, it remains true whenever the characteristic is 0; there is but one geometry of characteristic 0, to which the methods of the theory of analytic functions and of topology, analytic continuation, theta functions, the homology theory, etc., may legitimately be applied." To illustrate the technique, we will prove Proposition \ref{torsion-alg-closed}:

\begin{proposition}
\label{torsion-alg-closed-lefschetz}
Let $k$ be an algebraically closed field of characteristic zero and let $A/k$ be an abelian variety of dimension $g$. Then $A(k)[\tors]\cong (\bQ/\bZ)^{2g}$.
\end{proposition}
\begin{proof}
The data defining the abelian variety $A/k$ (i.e. the variety itself, the multiplication map, the inverse map, and the identity section) are all defined over a field $L$ which is finitely generated over $\bQ$. Any \emph{countably} generated extension of $\bQ$ can be embedded into $\bC$, so we can embed $L\subset\bar{L}\
\ra \bC$. Note that if $P\in A(k)$ is a point of order $n$, then in fact $P\in A(\bar{L})$. Now for each $n$, we have $A(k)[n]=A(\bar{L})[n]=A(\bC)[n]\isom (\bZ/n\bZ)^{2g}$ by Proposition \ref{torsion-C}. That $A(k)[\tors]\isom (\bQ/\bZ)^{2g}$ follows as before from purely group-theoretic considerations.
\end{proof}

There are many variants of the Lefschetz principle which can actually be ``proven" in a metamathematical sense (see \cite{barwise1969lefschetz}, for instance), but perhaps the simplest and most elegant is the following, due to Tarski \cite{tarski1951decision}: Let $k$ be an algebraically closed field of characteristic zero, and let $\phi$ be a first-order sentence in the language of rings $\mathcal{L}_r=\{0,1,+,-,\cdot\}$. Then $\phi$ holds in $\bC$ if and only if $\phi$ holds in $k$. In other words, this version of the Lefschetz principle states that the theory of algebraically closed fields is \emph{model complete}. One possible interpretation of the Lefschetz principle as stated by Weil is that the content of many theorems in algebraic geometry over an algebraically closed field $k$ of characteristic zero are reducible to some statements about $k$ itself in the language of rings where Tarski's version of the Lefschetz principle kicks in. But clearly not \emph{every} statement about algebraic geometry can be captured in this way: take, for instance, the statement that for every positive-dimensional variety $V/\bC$, the cardinality of $V(\bC)$ is equal to that of $\bC$. In general, it can often be difficult to decide whether the Lefschetz principle (in the interpretation above) is really a valid method of proof. For instance, does the Lefschetz principle handle Proposition \ref{torsion-alg-closed-lefschetz} above? Given an abelian variety $A/\bC$ of dimension $g$, it certainly seems \emph{plausible} that the claim ``$A(\bC)$ has exactly $n^{2g}$ points of order $n$" is equivalent to the existence of exactly $n^{2g}$ solutions to some system of polynomial equations with coefficients depending on $A$, but to actually justify such an equivalence is not so easy.

The impetus for this thesis is to give a formal justification for the use of the Lefschetz principle (interpreted as above), at least for the case of group-theoretic statements about the rational points of abelian varieties. Our technique is general enough to apply not only to algebraically closed fields of characteristic zero, but to any theory extending the theory of fields which is \emph{model complete} (e.g. real closed and $p$-adically closed fields). The key idea is that of an \emph{interpretation} of one structure in another.\footnote{The author would like to thank to Alex Kruckman for providing some helpful remarks on interpretation of structures}. This will require a brief detour in model theory.


\section{Model Theory}
Model theory is, in short, the mathematical study of the relationship between \emph{syntax} and \emph{semantics}, i.e. between language and meaning. Formal systems of language and theory are studied via their models, which are the structures which give meaning (and importantly, \emph{truth}) to the formal symbols of language. Model theory is quite a recent development in mathematics. Its philosophical roots can be traced back to late nineteenth- and early twentieth-century works of Frege and Russel and the mid-twentieth-century works of the French structuralists, but it was Tarski who developed model theory as a mathematical discipline in its own right in the 1950s.

The relationship between syntax and semantics is no stranger to the algebraic geometer. The fundamental duality of algebraic geometry is one of this nature: polynomial equations (syntax) and the corresponding set of solutions (semantics). One should not be surprised to learn of the many varied applications of model-theoretic ideas to mainstream algebraic geometry: look no further than this very paper! See also the recent and celebrated proof of the Andr\'e-Oort conjecture \cite{pila2022canonical}.

Let us begin with a definition:

\begin{definition}
A first-order language $\mathcal{L}$ is specified by the following data:
\begin{itemize}
	\item a set of function symbols $\mathcal{F}$ and for each $f\in \mathcal{F}$ a non-negative integer $n_f$.
	\item a set of relation symbols $\mathcal{R}$ and for each $R\in \mathcal{R}$ a non-negative integer $n_R$.
	\item a set of constant symbols $\mathcal{C}$.
\end{itemize}
The number $n_f$ (called the \emph{arity} of $f$) represents that $f$ is a function of $n$ variables. Similarly, the number $n_R$ represents that $R$ is an $n_R$-ary relation.
\end{definition}

\begin{example}
	\begin{enumerate}
		\item The language of groups $\mathcal{L}_g=\{\cdot, e\}$ consists of a binary function $\cdot$ and a constant symbol $e$.
		\item The language of rings $\mathcal{L}_r=\{+,-,\cdot, 0,1\}$ consists of the binary functions $+$ and $\cdot$, the unary function $-$, and the constant symbols $0$ and $1$. 
		\item The language of ordered rings $\mathcal{L}_{or} = \{+,-,\cdot,<,0,1\}$ appends to the language of rings a binary relation $<$.
		\item The language of valued fields $\mathcal{L}_{div}=\{+,-,\cdot,\mid, 0,1\}$ appends to the language of rings a binary relation $\mid$.
	\end{enumerate}
\end{example}

To give \emph{meaning} to the symbols in $\mathcal{L}$ is to give a so-called ``$\mathcal{L}$-structure":

\begin{definition}
An $\mathcal{L}$-structure $\mathcal{M}$ is specified by the following data:
\begin{itemize}
	\item A set $M$, called the \emph{domain} of $\mathcal{M}$
	\item a function $f^{\mathcal{M}}:M^{n_f}\ra M$ for each function symbol $f\in\mathcal{F}$ of arity $n_f$
	\item a relation $R^{\mathcal{M}}\subset M^{n_R}$ for each relation symbol $R\in\mathcal{R}$ of arity $n_R$.
	\item an element $c^{\mathcal{M}}\in M$ for each constant symbol $c\in\mathcal{C}$.
\end{itemize}
$f^{\mathcal{M}}$, $R^{\mathcal{M}}$, and $c^{\mathcal{M}}$ are called \emph{interpretations} of the symbols $f, R$, and $c$ in $M$. The Roman letters $A,B,C,\dots,$ always refer to the underlying universe of the $\mathcal{L}$-structure $\mathcal{A},\mathcal{B},\mathcal{C},\dots$.
\end{definition}

\begin{example}
	\begin{itemize}
		\item Any group is an $\mathcal{L}_g$-structure, with the symbols interpreted in the obvious ways.
		\item Any ring, or even a field, is an $\mathcal{L}_r$-structure.
		\item The presence of extra structure in the previous examples is somewhat misleading. For instance, the natural numbers $\bN$ with the binary operation $n\cdot m = n^{3m}+20$ and distinguished element $e=21421$ is a perfectly valid $\mathcal{L}_g$-structure.
	\end{itemize}
\end{example}

Of course, one can speak of a homomorphism of $\mathcal{L}$-structures:

\begin{definition}
A homomorphism $\phi:\mathcal{M}\ra \mathcal{N}$ of $\mathcal{L}$-structures is a function $\phi:M\ra N$ that ``preserves interpretations" in the following sense:
\begin{enumerate}
	\item for every $n$-ary function symbol $f\in\mathcal{F}$, we have $\phi(f^{\mathcal{M}}(a_1,\dots, a_n))=f^{\mathcal{N}}(\phi(a_1),\dots, \phi(a_n))$ for all $a_1,\dots, a_n\in M$.
	\item for every $n$-ary relation symbol $R\in\mathcal{R}$, we have $(a_1,\dots, a_n)\in R^{\mathcal{M}}$ only if $(\phi(a_1),\dots, \phi(a_n))\in R^{\mathcal{N}}$.
	\item for every constant symbol $c\in\mathcal{C}$, we have $\phi(c^{\mathcal{M}})=c^{\mathcal{N}}$
\end{enumerate}
If, furthermore, for every $n$-ary relation symbol $R\in\mathcal{R}$, we have $(a_1,\dots, a_n)\in R^{\mathcal{M}}$ \emph{if and only if} $(\phi(a_1),\dots, \phi(a_n))\in R^{\mathcal{N}}$, then we say $\phi$ \emph{reflects} $\mathcal{R}$. An injective homomorphism $\phi$ which reflects $\mathcal{R}$ is called an \emph{embedding}; a surjective embedding is called an \emph{isomorphism}.
\end{definition}

We can use the language $\mathcal{L}$ together with logic symbols (variables $x_1,x_2,x_3,\dots$, equality $=$, Boolean connectives $\top, \bot, \wedge, \vee, \neg, \implies, \iff$, existential and universal quantifiers $\forall$ and $\exists$, and parentheses) to create \emph{formulas} that describe certain properties of $\mathcal{L}$-structures. For example, an abelian group $A$ is an $\mathcal{L}_g$-structure for which the (interpretation of the) formula $\forall x \forall y( x\cdot y = y\cdot x)$ is true. We proceed by formalizing the intuitive notion of a formula:

\begin{definition}
The set of $\mathcal{L}$-terms is the smallest set $\mathcal{T}$ such that 
\begin{enumerate}
	\item $c\in \mathcal{T}$ for each constant symbol $c\in\mathcal{C}$
	\item $x_i\in \mathcal{T}$ for each variable $x_i$, $i=1,2,3,\dots$
	\item If $t_1,\dots, t_n\in\mathcal{T}$ and $f\in\mathcal{F}$ is an $n$-ary function symbol, then $f(t_1,\dots, t_n)\in\mathcal{T}$
\end{enumerate} 
\end{definition}

This definition is recursive, giving us a method for proving a statement about all terms: prove the statement for each variable and constant symbol, and in the inductive step, assume it holds true for terms $t_1,\dots, t_n$, then prove it holds for $f(t_1,\dots, t_n)$.

Given a term and an $\mathcal{L}$-structure, there is a natural way of ``evaluating" the term in $\mathcal{M}$:

\begin{definition}
Let $\mathcal{M}$ be an $\mathcal{L}$-structure. Let $t$ be a term in the variables $(x_1,\dots, x_n)$ and let $\bar{a}=(a_1,\dots, a_n)\in M^n$. We define the \emph{evaluation} of $t$ at $\bar{a}$ inductively in the complexity of $t$:
\begin{itemize}
	\item if $t$ is the variable $x_i$, then $t^{\mathcal{M}}(\bar{a})=a_i$
	\item if $t$ is a constant symbol $c$, then $t^{\mathcal{M}}(\bar{a})=c^{\mathcal{M}}$
	\item if $t$ is a compositie term $f(t_1,\dots, t_n)$ and we have elements $t_i^{\mathcal{M}}(\bar{a})$ for each $i=1,\dots, n$, then $t^{\mathcal{M}}(\bar{a})=f^{\mathcal{M}}(t_1^{\mathcal{M}}(\bar{a}),\dots, t_n^{\mathcal{M}}(\bar{a}))$.
\end{itemize}
\end{definition}

We are now ready to give the definition of an $\mathcal{L}$-formula:

\begin{definition}
An \emph{atomic }$\mathcal{L}$\emph{-formula} is either
\begin{itemize}
	\item $t_1=t_2$, where $t_1$ and $t_2$ are terms
	\item $R(t_1,\dots, t_n)$, where $R\in\mathcal{R}$ is an $n$-ary relation and $t_1,\dots, t_n$ are terms.
\end{itemize}
An $\mathcal{L}$\emph{-formula} is one of the following:
\begin{itemize}
	\item $\top$ or $\bot$
	\item an atomic $\mathcal{L}$-formula
	\item $\phi\vee \psi$, $\phi\wedge \psi$, or $\neg\phi$, where $\phi$ and $\psi$ are formulas
	\item $\exists x \hspace{0.1cm}\phi(x)$ or $\forall x \hspace{0.1cm}\phi(x)$, where $\phi$ is a formula
\end{itemize}
A formula with no free variables is called a \emph{sentence}.
\end{definition}

Again, this definition is recursive so that we obtain a method of proof by induction on the complexity of $\mathcal{L}$-formulae.

\begin{example}
	\begin{itemize}
		\item $((x\cdot x)\cdot y)\cdot y=e$ is a formula in the language of groups. The formal syntax described above is quite cumbersome, so in practice we often revert to our natural notation, where the above formula would be equivalently represented by $x^2y^2=e$.
		\item $\exists x(x^2+2x+1>0)$ is a formula in the language of ordered rings
		\item $\forall x\forall y(\neg(xy\mid z)\vee \exists w(zw=1 \wedge 0=1))$ is a formula in the language of valued fields.
	\end{itemize}
\end{example}

Given an $\mathcal{L}$-sentence $\phi$ and an $\mathcal{L}$-structure $\mathcal{M}$, we are now equipped with the tools to speak of the \emph{truth} $\phi$ in $\mathcal{M}$:

\begin{definition}
Let $\mathcal{M}$ be an $\mathcal{L}$-structure. Let $\phi$ be an $\mathcal{L}$-formula in the variables $(x_1,\dots,x_n)$ and $\bar{a}=(a_1,\dots,a_n)\in M^n$. We define the relation $\mathcal{M}\models \phi(\bar{a})$ (or simply $M\models \phi(\bar{a})$), read $M$ \emph{satisfies} $\phi(\bar{a})$, by induction on the complexity of $\phi$:
	\begin{itemize}
		\item if $\phi$ is $t_1=t_2$, then $M\models\phi(\bar{a})$ iff $t_1^{\mathcal{M}}(\bar{a})=t_2^{\mathcal{M}}(\bar{a})$
		\item if $\phi$ is $R(t_1,\dots, t_n)$ then $M\models\phi(\bar{a})$ iff $(t_1^{\mathcal{M}}(\bar{a}),\dots, t_n^{\mathcal{M}}(\bar{a}))\in R^{\mathcal{M}}$
		\item if $\phi$ is $\top$, then $M\models\phi(\bar{a})$
		\item if $\phi$ is $\bot$, then $M\not\models\phi(\bar{a})$
		\item if $\phi$ is $\psi\vee \chi$, then $M\models \phi(\bar{a})$ iff $M\models \psi(\bar{a})$ or $M\models \chi(\bar{a})$
		\item if $\phi$ is $\psi\wedge \chi$, then $M\models \phi(\bar{a})$ iff $M\models \psi(\bar{a})$ and $M\models \chi(\bar{a})$
		\item if $\phi$ is $\neg\psi$, then $M\models \phi(\bar{a})$ iff $M\not\models \psi(\bar{a})$
		\item if $\phi$ is $\exists y\hspace{0.1cm}\psi(y)$, then $M\models \phi(\bar{a})$ iff there exists $b\in M$ such that $M\models \psi(\bar{a},b)$
		\item if $\phi$ is $\forall y\hspace{0.1cm}\psi(y)$, then $M\models \phi(\bar{a})$ iff for all $b\in M$, $M\models \psi(\bar{a},b)$
	\end{itemize}
\end{definition}

\begin{example}
	\begin{itemize}
		\item Let $\phi$ be the $\mathcal{L}_r$-formula $\exists x (x^2=-1)$. Then $\bC\models \phi$, while $\bR\not\models \phi$
		\item Let $\phi$ be the $\mathcal{L}_g$-formula $\forall x\forall y(xy=yx)$. The abelian groups are precisely the groups $A$ for which $A\models \phi$. (Of course, there are also $\mathcal{L}_g$-structures which are not groups at all which still satisfy $\phi$: consider $\bN$ with the binary operation $+$ and distinguished element $10$).
		\item Let $\phi(y)$ be the $\mathcal{L}_r$-formula $\exists x(x^2=y)$. Then $\bR\models\phi(2)$, while $\bQ\not\models\phi(2)$.
	\end{itemize}
\end{example}

Given a fixed $\mathcal{L}$-formula $\phi(x_1,\dots, x_n)$ and an $\mathcal{L}$-structure $\mathcal{M}$, one can study the subset of tuples $(a_1,\dots, a_n)\in M^n$ for which $M\models \phi(a_1,\dots, a_n)$. The study of such subsets is the cornerstone of model theory:

\begin{definition}
Let $\mathcal{M}$ be an $\mathcal{L}$-structure. We say that a subset $X\subset M^n$ is \emph{definable} if there is an $\mathcal{L}$-formula $\phi(x_1,\dots, x_n, y_1,\dots, y_m)$ and a tuple $\bar{b}=(b_1,\dots, b_m)\in M^m$ such that $$X=\{\bar{a}\in M^n: M\models \phi(\bar{a},\bar{b})\}.$$ The elements of the tuple $\bar{b}$ are called the \emph{parameters} of the definable set. 
\end{definition}

\begin{example}
Classical (affine) algebraic varieties are definable in the language $\mathcal{L}_r$. Indeed, if $V$ is the common zero-locus of the ideal of functions $I=(f_1,\dots, f_n)\subset k[x_1,\dots,x_m]$, then $V$ is defined by the formula $$\bigwedge_{i=1}^n f_i(x_1,\dots, x_n)=0.$$ Note that this is a formula with parameters given by the coefficients of the $f_i$. As a concrete example, the variety $V\subset\bR^2$ cut out by the equation $x^2-\pi y^2+1=0$ is defined by the $\mathcal{L}_r$-formula $\phi(x,y,z): x^2-zy^2+1=0$ with parameter $z=\pi$, i.e. $$V=\{(a,b)\in\bR^2: \bR\models\phi(a,b,\pi)\}.$$
\end{example}

Given a language $\mathcal{L}$, we most often consider not just arbitrary $\mathcal{L}$-structures, but rather $\mathcal{L}$-structures that satisfy certain extra requirements, and typically these extra requirements can be formulated themselves in the given language. For example, the axioms of a group are all sentences in the language of groups $\mathcal{L}_g$. This leads us to the following

\begin{definition}
An $\mathcal{L}$-\emph{theory} $T$ is a set of $\mathcal{L}$-sentences (recall that a \emph{sentence} is a formula with no free variables). A $\mathcal{L}$-structure $\mathcal{M}$ is called a \emph{model} of $T$ if $M\models \phi$ for all $\phi\in T$. If $\phi$ is any sentence, then we say $T$ \emph{entails} $\phi$, written $T\models \phi$ if every model of $T$ satisfies $\phi$.
\end{definition}

\begin{example}
	\begin{enumerate}
		\item The theory of groups $T_g$ consists of the $\mathcal{L}_g$-sentences\vspace{-.2cm}
			\begin{center}
			$\forall x\forall y\forall z (x\cdot(y\cdot z) = (x\cdot y)\cdot z)$\\
			$\forall x (x\cdot e = x\wedge e\cdot x = x)$\\
			$\forall x \exists y (x\cdot y = e\wedge y\cdot x = e)$
			\end{center}
$T_g\models \forall x\forall y\forall z (x\cdot y = x\cdot z\implies y=z)$ since every group has left cancellation. On the other hand, $T_g\not\models \forall x\forall y (x\cdot y = y\cdot x)$ since there are non-abelian groups.
		\item The theory of algebraically closed fields of characteristic zero $\catname{ACF}_0$ consists of the field axioms together with the $\mathcal{L}_r$-sentences $$\forall a_0\dots \forall a_{n-1}\exists x (x^n + \sum_{i=0}^{n-1}a_i x^i=0)$$ for each $n=1,2,\dots$ and the sentences $p\neq 0$ for each prime $p$.
	\end{enumerate}
\end{example}

\begin{definition}
An $\mathcal{L}$-theory $T$ is said to be \emph{model complete} if it is satisfiable (i.e. it has a model) and for every $\mathcal{L}$-sentence $\phi$, either $T\models \phi$ or $T\models \not\phi$. 
\end{definition}

We have just seen that the theory of groups is not model complete. On the other hand, the theory of algebraically closed fields of characteristic zero turns out to be model complete. We will also see later that the theory of real closed fields and the theory of $p$-adically closed fields are both model complete.

The most useful characterization of model complete theories is that an $\mathcal{L}$-theory $T$ (for our purposes) is that any two models of $T$ are \emph{elementarily equivalent}, i.e. if $\mathcal{M}$ and $\mathcal{N}$ are models of $T$ and if $\phi$ is any $\mathcal{L}$-sentence, then $M\models \phi$ if and only if $N\models \phi$.

We have thus set up the necessary language to describe an \emph{interpretation} of one structure in another (not to be confused with the interpretation of the symbols in an $\mathcal{L}$-structure):

%interpretation%
\begin{definition}
An interpretation of an $\mathcal{L}$-structure $\mathcal{M}$ in an $\mathcal{L}'$-structure $\mathcal{N}$ consists of
\begin{itemize}
\item an $\mathcal{L}'$-definable set $D\subset N^n$,
\item an $\mathcal{L}'$-definable equivalence relation $\sim$ on $D$
\item for each $k$-ary function symbol in $\mathcal{L}$, an $\mathcal{L}'$-definable function $(D/\sim)^k\ra (D/\sim)$,
\item for each $k$-ary relation symbol in $\mathcal{L}$, an $\mathcal{L}'$-definable subset of $(D/\sim)^k$
\item for each constant symbol in $\mathcal{L}$, an element of $(D/\sim)$
\end{itemize}
such that the resulting $\mathcal{L}$-structure $\mathcal{D}$ with domain $D/\sim$ is isomorphic to $\mathcal{M}$.
\end{definition}

As one might expect, when an $\mathcal{L}$-structure $\mathcal{M}$ is interpretable in an $\mathcal{L}'$-structure $\mathcal{N}$, one can translate $\mathcal{L}$-formulas into $\mathcal{L}'$-formulas in a truth-preserving way:

\begin{proposition}
Let $\mathcal{M}$ be an $\mathcal{L}$-structure which is interpretable in an $\mathcal{L}'$-structure $\mathcal{N}$ and let $\mathcal{D}$ be the $\mathcal{L}'$ For any first-order $\mathcal{L}$-formula $\phi(x_1,\dots, x_k)$, there is a first-order $\mathcal{L}'$ formula $\psi(y_1,\dots, y_k)$ (where each $y_i$ is an $n$-tuple of variables) such that for all $a_1,\dots, a_k\in D$, $N\models \psi(a_1,\dots, a_k)$ if and only if $M\models \phi(f([a_1]_{\sim}), \dots, f([a_k]_{\sim}))$, where $f:\mathcal{D}\ra \mathcal{M}$ is the isomorphism.
\end{proposition}
\begin{proof}
This is a simple proof by induction on the complexity of $\phi$. NOTE: perhaps I should prove this to demonstrate the technique?
\end{proof}

\begin{example}
	\begin{itemize}
		\item One can interpret the $\mathcal{L}_r$-structure $\bQ$ in the $\mathcal{L}_r$-structure $\bZ$ as follows (in fact, this is typically how $\bQ$ is \emph{defined}): let $D\subset \bZ^2$ be the definable subset given by the formula $\phi(x,y): y\neq 0$ and consider the equivalence relation $(a,b)\sim (c,d)$ iff $ad=bc$; addition, subtraction, and multiplication are defined on $(D/\sim)$ in the familiar way, e.g. $[(a,b)]_{\sim}+[(c,d)]_{\sim} = [(ad+bc, bd)]_{\sim}$; 0 and 1 are represented by the classes $[(0,1)]_{\sim}$ and $[(1,1)]_{\sim}$, respectively. Then the homomorphism $f:(D/\sim)\ra \bQ$ given by $f([(a,b)]_{\sim})=a/b$ is an isomorphism of $\mathcal{L}_r$-structures. Now consider the $\mathcal{L}_r$-sentence $\phi$ given by $\forall x\forall y(x+y=y+x)$. Our interpretation of $\bQ$ in $\bZ$ allows us to ``translate" $\phi$ into the $\mathcal{L}_r$-sentence $\psi$ given by $$\forall a\forall b\forall c\forall d((b\neq 0\and d\neq 0)\implies (ad+bc)db=bd(cb+da))$$ 
		\item Consider the affine group scheme $G=\Spec(k[x]/(x^2-1))$, where $k$ is a field of characteristic not equal to 2. Its group of $k$-points constitute an $\mathcal{L}_g$-structure (isomorphic to $\bZ/2\bZ$). We can use the algebraic-geometry-commutative-algebra duality to interpret this group of $k$-points as an $\mathcal{L}_r$-structure in the field $k$: let $D\subset k$ be the definable subset given by the formula $\phi(x): x^2=1$ and let $\sim$ be the equivalence relation given just by equality. We interpret the group operation on $G(k)$ by multiplication in $k$ and we interpret the identity in $G(k)$ by $1\in D$. Now consider the $\mathcal{L}_g$-sentence $\phi$ given by $\forall x\forall y(xy=yx)$. Again, our interpretation of $G(k)$ in $k$ allows us to ``translate" this sentence into the $\mathcal{L}_r$-sentence $\psi$ given by $$\forall a\forall b((a^2=1\wedge b^2=1)\implies ab=ba)$$
		\item Let's go for a more ambitious and illustrative example. Consider the elliptic curve $E/\bR$ given in Weierstrass form by $y^2=x^3+\pi$. Its group of $\bR$-points constitute an $\mathcal{L}_g$-structure and we can interpret it in the $\mathcal{L}_r$-structure $\bR$: let $D\subset\bR^3$ be the subset defined by $\phi(x,y,z,\pi)$ given by $$(y^2z=x^3+\pi z^3)\wedge (x\neq 0 \vee y\neq 0 \vee z\neq 0)$$ and consider the equivalence relation $(a_0,a_1,a_2)\sim (b_0,b_1,b_2)$ iff $\exists\lambda(\lambda\neq 0\wedge\bigwedge_{i=0}^2 a_i=\lambda b_i$. The class $[(a,b,c)]_{\sim}$ will be referred to by $[a:b:c]$, as is typical with projective coordinates. The identity element is interpreted by $[0:1:0]$ and the group operation is interpreted by the function $[x_0:y_0:z_0]\cdot[x_1:y_1:z_1]=[x_2:y_2:z_2]$ whose graph is defined $\phi: \phi_1\vee\phi_2\vee (\phi_3\wedge (\phi_4\vee \phi_5\vee \phi_6)))$, where
		\begin{align*}
		\phi_1 &: (z_0=0)\wedge (x_2=x_1)\wedge (y_2=y_1)\wedge (z_2=z_1)\\
		\phi_2 &: (z_1=0)\wedge (x_2=x_0)\wedge (y_2=y_0)\wedge (z_2=z_0)\\
		\phi_3 &: \exists u\exists v((uz_0=1)\wedge (vz_1=1))\\
		\phi_4 &: (x_0u=x_1v)\wedge (y_0u+y_1v=0)\wedge (x_2=0)\wedge (y_2=1)\wedge (z_2=0)\\
		\phi_5 &: (x_0u=x_1v)\wedge \exists s ((2y_0us=1)\wedge (x_2=(3(x_0u)^2s)^2-x_0u-x_1v)\wedge\\& (y_2=-3(x_0u)^2s((3(x_0u)^2s)^2-x_0u-x_1v)-(-(x_0u)^3+2\pi)s))\wedge (z_2=1))\\
		\phi_6 &: \exists t (t(x_1v-x_0u)=1\wedge (x_2= (t(y_1v-y_0u))^2-x_0u-x_1v)\wedge\\& (y_2=-t(y_1v-y_0u)((t(y_1v-y_0u))^2-x_0u-x_1v)-t(y_0ux_1v-y_1vx_0u))\wedge (z_2=1))
		\end{align*}

		Note that this is simply a projectivized and logicized version of the algorithm given in Silverman \cite{silverman2009arithmetic}[Group Law Algorithm 2.3] when $a_1=a_2=a_3=a_4=0$ and  $a_6=\pi$. We can translate the $\mathcal{L}_g$-sentence $\phi$ given by \begin{equation}
		\exists x\exists y (x^2=e\wedge y^2=e\wedge\forall z(z^2=e\implies (z=x\vee z=y)))
		\end{equation}
		into an equivalent $\mathcal{L}_r$-formula $\psi$ given by \begin{equation}\psi_1\wedge\psi_2\wedge (\psi_3\implies(\psi_4 \vee \psi_5))
		\end{equation}
		where
		\begin{align*}
		\psi_1 &: \exists x_0\exists y_0\exists z_0\exists x_1\exists y_1\exists z_1((y_0^2z_0=x_0^3+\pi z_0^3)\wedge (y_1^2z_1=x_1^3+\pi z_1^3))\\
		\psi_2 &: \exists \lambda_0\exists\lambda_1(\lambda_0\neq 0\wedge\lambda_1\neq 0 \wedge\phi(x_0,y_0,z_0,x_0,y_0,z_0,0,\lambda_0,0)\\&\wedge\phi(x_1,y_1,z_1,x_1,y_1,z_1,0,\lambda_1,0)\\
		\psi_3 &: \forall x_2\forall y_2\forall z_2 ((y_2^2z_2=x_2^3+\pi z_2^3) \wedge \exists\lambda_2(\lambda_2\neq 0\wedge\phi(x_2,y_2,z_2,x_2,y_2,z_2,0,0,1,0)))\\
		\psi_4 &: \exists s (s\neq 0\wedge x_2=sx_0\wedge y_2=sy_0\wedge z_2=sz_0)\\
		\psi_5 &: \exists t(t\neq 0\wedge x_2=tx_1\wedge y_2=ty_1\wedge z_2=tz_1)
		\end{align*}
		NOTE: Is there a better way to display all of this? It's quite horrible, but given the complexity of the interpretation I think it's unavoidable...
	\end{itemize}
\end{example}

The group of $k$-points of an abelian variety is always interpretable in $k$:

\begin{proposition}
Let $k$ be a field and $A/k$ an abelian variety. Then the group $A(k)$ is interpretable in the field $k$.
\end{proposition}
\begin{proof}
This is well-known. See, for example, Pillay's article in \cite{bouscaren2009model}. The example worked out above illustrates the general procedure: abelian varieties are projective, so one may interpret the $k$-points of $A$ as solutions to a system of homogeneous polynomials in the projective space $\bP^n(k)$, which is the quotient of a definable subset of $k^{n+1}$ by a definable equivalence relation. The addition map is a regular morphism, meaning that there are finitely many affine open subsets on which the morphism is given coordinate-wise by the ratio of two homogeneous polynomials of the same degree.
\end{proof}

We can, in effect, rid ourselves of the pesky parameters that arise out of interpreting a single abelian variety by interpreting group-theoretic statements uniformly across definable families:

\begin{theorem}
Let $\mathcal{A}\ra S$ be a quasi-projective abelian scheme with $S=\Spec(R)$, where $R$ is of finite type over $\bZ$. Let $k$ be a model of an $\mathcal{L}$-theory $T$ which extends the theory of fields and let $M\subset S(k)$ an $\mathcal{L}$-definable subset without parameters given by a formula $\rho$. For any sentence $\phi$ in the language of groups, there is an $\mathcal{L}$-formula $\psi$ so that if $\mathbf{a}\in M$, then $\mathcal{A}_{\mathbf{a}}(k)\models\phi$ if and only if $k\models\psi(\mathbf{a})$, where $\mathcal{A}_{\mathbf{a}}/k$ is the fiber of $\mathcal{A}$ over the $k$-point of $S$ corresponding to $\mathbf{a}$. In particular, $\mathcal{A}_{\mathbf{a}}(k)$ satisfies $\phi$ for all $\mathbf{a}\in M$ if and only if $k$ satisfies the sentence $\chi = \forall \mathbf{a}(\rho(\mathbf{a})\implies \psi(\mathbf{a}))$.
\end{theorem}
\begin{proof}
NOTE: I've suppressed all the variable contexts in the formulation of the theorem, e.g. I've written just $\rho$ in place of $\rho(x_1,\dots, x_n)$. Is this okay?
The first statement is clear from the above proposition. The second statement follows since the data defining $\mathcal{A}$ is \emph{uniform} in the parameters of $S$, for, at bottom, that is what is \emph{meant} by an abelian scheme.
\end{proof}

\begin{example}
Consider the family of elliptic curves $y^2=x^3+t$ with $t\neq 0$. Its group of $\bR$-points can be interpreted exactly as the previous example illustrates with the parameter $t\neq 0$ in place of $\pi$. The $\mathcal{L}_g$-sentence $\phi$ given above in equation 1.1 can be interpreted uniformly across the whole family by the $\mathcal{L}_r$-sentence $\chi$ given by $$\forall t (t\neq 0\implies \psi(t)),$$ where $\psi$ is given above in equation 1.2, but $\pi$ replaced by the parameter $t$. The fact that every real elliptic curve of the form $y^2=x^3+t$ has exactly two real 2-torsion points is now expressed symbolically by $\bR\models\chi$.
\end{example}

We are now ready to formula our Lefschetz-like transfer principle:

\begin{theorem}
\label{transfer-theorem}
Let $\mathcal{L}$ be a language extending the language of rings, $T$ a model complete $\mathcal{L}$-theory extending the theory of fields, and $K$ and $L$ two models of $T$. Let $\mathcal{A}/S$ be as in the above theorem. Let $\rho$ be an $\mathcal{L}$-formula such that $M:=\{\mathbf{a}:K\models \rho(\mathbf{a})\}\subset S(K)$ and $N:=\{\mathbf{b}:L\models \rho(\mathbf{b})\}\subset S(L)$. Let $\phi$ be a sentence in the language of groups. Then $A_{\mathbf{a}}(K)$ satisfies $\phi$ for all $K$-points $\mathbf{a}\in M$ if and only if $A_{\mathbf{b}}(L)$ satisfies $\phi$ for all $L$-points $\mathbf{b}\in M$.
\end{theorem}
\begin{proof}
By the above theorem, there is an $\mathcal{L}$-formula $\psi$ so that $A_{\mathbf{a}}(K)$ satisfies $\phi$ for all $\mathbf{a}\in M$ iff and only if $K\models \psi$. Since $T$ is model complete, this is the case if and only if $L\models \psi$, which in turn is the case if and only if $A_{\mathbf{b}}(L)$ satisfies $\phi$ for all $\mathbf{b}\in N$.
\end{proof}

\begin{example}
Let us return the family of elliptic curves $\mathcal{E}/S$ from Section 1.1. Now let $k$ be any algebraically closed field of characteristic zero. As previously discussed, the fibers over the $k$-valued points of $S$ constitute all elliptic curves over $k$, up to isomorphism. Consider the $\mathcal{L}_g$-sentence $\phi$ given by $$\exists a_1\dots\exists a_{n^2}\left(\bigwedge_{i=1}^{n^2} a_i^{n}=e\wedge \forall b(b^n=e\implies \bigvee_{i=1}^{n^2}b=a_i)\right),$$ which says that there are exactly $n^2$ points killed by $n$. Proposition \ref{torsion-C} tells us that for every complex elliptic curve $A/\bC$, $A(\bC)$ satisfies $\phi$. Thus, taking $\rho$ to be the sentence $\forall a\forall b(4a^3+27b^2\neq 0)$, the above Theorem \ref{transfer-theorem} tells us that every elliptic curve $E/k$ also satisfies $\phi$.
\end{example}

The final step is to prove that there are enough definable families to actually make use of this transfer principle for higher dimensional abelian varieties.

\begin{lemma}
\label{comprehensive-family}
For every pair of positive integers $g$ and $d$, there is a finite collection of abelian schemes $A_i\rightarrow S_i$ whose bases $S_i$ are quasi-projective (and therefore finite type) over $\Spec(\bZ)$ that satisfy the following condition: for every field $K$ and abelian variety $B/K$ of dimension $g$ possessing a polarization of degree $d^2$, there is a $K$-point $s:\Spec(K)\rightarrow S_i$ for some $i$ so that the fiber $(A_i)_s/K$ is isomorphic to $B$ as an abelian variety over $K$.
\end{lemma}
\begin{proof} For any fixed rational polynomial $\Phi$, recall that the functor $$\Hilbit(n,\Phi):\Sch^{op}\ra \Set$$ given by 
$$\Hilbit(n,\Phi)(T)=\{Y\subset \mathbb{P}^n_T: Y\subset \mathbb{P}^n_T\text { is closed with Hilbert polynomial } \Phi\\ \text{ and }Y/T\text{ is flat}\}$$ is represented by a scheme $\Hilb(n,\Phi)$ which is projective over $\Spec(\bZ)$ (see \cite{FGAHilbert}, for instance). Let $Z/\Hilb(n,\Phi)$ be the universal object of this functor. Then $Z$ represents the functor $\Hilbit^*(n,\Phi):\Sch^{op}\rightarrow \Set$ which further specifies a section:
\begin{align*}
&\Hilbit^*(n,\Phi)(T)=\{(Y\subset \mathbb{P}^n_T, s: T\rightarrow Y) : Y\subset \mathbb{P}^n_T\text { is closed with Hilbert} \\&\text{ polynomial } \Phi,Y/T\text{ is flat, and } s\text{ is a section}\}
\end{align*}

\noindent Let $(Z'/Z, \varepsilon: Z\ra Z')$ be the universal object for this functor. Now, given an abelian variety $B/K$ of dimension $g$ possessing a polarization $\phi:B\ra B^{\vee}$ of degree $d^2$, the line bundle $(\id,\phi)^*(\mathcal{P})^3$ (where $\mathcal{P}$ is the Poincar\'e bundle on $B\times B^{\vee}$) induces an embedding $B\hookrightarrow \bP^n_K$ for which $n$ and the Hilbert polynomial $\Phi$ of $B$ are determined entirely by $g$ and $d$ (see \cite[Ch. 3, sec. 16]{mumfordAV}). We thus obtain a morphism $\Spec(K)\rightarrow Z$ (since abelian varieties come with sections). After restricting to the irreducible components of $Z$ and considering the smooth locus of $Z'\ra Z$, \cite[Theorem 6.14]{GIT} ensures that we have a finite collection of abelian schemes with the desired property.
\end{proof}

Let's illustrate our new technique by giving another proof of Proposition \ref{torsion-alg-closed-lefschetz}:

\begin{proposition}
Let $k$ be an algebraically closed field of characteristic zero and let $A/k$ be an abelian variety of dimension $g$. Then $A(k)[\tors]\cong (\bQ/\bZ)^{2g}$.
\end{proposition}
\begin{proof}
\label{torsion-alg-closed-transfer}
$A$ possess a polarization of degree $d^2$ for some $d$, so by Lemma \ref{comprehensive-family}, there is an abelian scheme $\mathcal{A}/S$ with $S$ of finite type over $\Spec(\bZ)$, and a $k$-point $s:\Spec(k)\ra S$ so that the fiber $\mathcal{A}_s/k$ is isomorphic to $A$ as an abelian variety over $k$. Without loss of generality, we may assume that $S$ is affine. The statement ``there are exactly $n^{2g}$ elements killed by $n$" is a first order statement in the language of groups which, by Proposition \ref{torsion-C}, is known to hold for the fiber over each $\bC$-point of $S$. The theory of algebraically closed fields of characteristic zero is model complete, so by Theorem \ref{transfer-theorem}, it also holds for the fiber over every $k$-point of $S$. In particular, it holds for $A$. Again, that $A(k)[\tors]$ has the given form now follows from purely group-theoretic considerations.
\end{proof}
\externaldocument{intro}

\chapter{Abelian varieties over real closed fields}
The first application of our transfer principle will be to the case of abelian varieties over real closed fields. We begin with a detailed discussion of real closed fields.

\section{Real closed fields}
The study of real closed fields largely began in an effort to solve Hilbert's 17th problem: 
\begin{center}
given $f\in\bR[x_1,\dots,x_n]$ with $f(a_1,\dots, a_n)\geq 0$ for all $(a_1,\dots,a_n)\in\bR^n$, can $f$ be expressed as a sum of squares of rational functions? i.e. does there exists $g_1,\dots, g_k\in \bR(x_1,\dots, x_n)$ such that $f=\sum_{i=1}^k g_i^2$?
\end{center}

One must allow for rational functions, as the polynomial $$f(x,y,z)=z^6+x^4y^2+x^2y^4-3x^2y^2z^2$$ takes only non-negative values, but cannot be written as a sum of squares of polynomials \cite{roy2000role}.

In 1927, Emil Artin demonstrated a positive solution to this problem. His proof has the rather striking feature that one must consider not only points of $\bR^n$, but points that belong to certain large extensions of the field of rational functions $\bR(x_1,\dots,x_n)$; these extensions satisfy certain algebraic properties analogous to those of the real numbers and as such were coined \emph{real closed fields}.

\begin{definition}
	\begin{enumerate}
		\item An ordered field $(R,\leq)$ is a field $R$ together with a linear order $\leq$ on $R$ such that for all $a,b,c\in R$ 
			\begin{itemize}
				\item if $a\leq b$, then $a+c\leq b+c$ and
				\item if $a\leq b$ and $0\leq c$, then $ac\leq bc$
			\end{itemize}
		\item A field $R$ is said to be \emph{real closed} if there is a unique linear ordering $\leq$ on $R$ making $(R,\leq)$ into an ordered field and such that every positive element has a square root in $R$ and every odd-degree polynomial has a root in $R$.
	\end{enumerate}
\end{definition}

\begin{example}
	\begin{itemize}
		\item The real numbers $\bR$ are real closed. 
		\item The field of real algebraic number $\bar{\bQ}\cap \bR$ is real closed.
		\item The field of Puiseux series $R\langle\langle T\rangle\rangle=\bigcup_{n\geq 1} \bR((T^{1/n}))$ is real closed. It's ordering is that induced by the valuation $v(f)=k/N$, where $f=\sum\limits_{n=k}^{\infty} a_nT^{n/N}$ and $a_k\neq 0$.
		\item The field of computable numbers --- those real number which can be computed to arbitrary precision in finite time by a Turing machine--- is real closed: the Taylor series for $\sqrt{x}$ can be used to calculate square roots and Newton's method can be used to calculate roots of odd-degree polynomials; both methods can easily be implemented on Turing machines.
	\end{itemize}
\end{example}

Surprisingly, one can determine the orderability of a field through purely field-theoretic considerations. This seemingly simple fact turns out to be quite important in developing the theory of real closed fields.

\begin{definition}
A field $K$ is said to be \emph{formally real} if $-1$ is not a sum of squares in $K$.
\end{definition}
\begin{theorem}
\label{formallyreal-orderability}
A field $K$ is formally real if and only if there is a linear order $\leq$ on $K$ such that $(K,\leq)$ is an ordered field. Moreover, if $a\in K$ is not a sum of squares in $K$, then we can choose an ordering such that $a\leq 0$. 
\end{theorem}
\begin{proof}
It is immediate that an ordered field is formally real. The converse is part of a famous result of Artin and Schreier cited below (Theorem \ref{artin-schreier}).
\end{proof}

Given an ordered field $(F,\leq)$, one can consider ordered extensions of $F$ which extend its ordering. Chasing the analogy with algebraically closed fields, one is led to consider ``maximal" such extensions:

\begin{definition}
Let $F$ be a formally real field. A formally real field $R$ is a \emph{real closure} of $F$ if $R/F$ is algebraic and $R$ is maximal among formally real algebraic extensions of $F$.
\end{definition}

\begin{proposition}
Let $F$ be a formally real field. Then $F$ has a real closure and this real closure is real closed.
\end{proposition}
\begin{proof}
Let $K$ be an algebraic closure of $F$ and consider the set $$\mathcal{R}:=\{R: F\subset R\subset K\text{ and } R \text{ is formally real} \}.$$ $\mathcal{R}$ is poset under inclusion. One easily proves that if $(R_i)_i$ is a chain in $\mathcal{R}$ that $\bigcup_i R_i\subset K$ is itself formally real and thus an upper bound of $(R_i)_i$ in $\mathcal{R}$. Zorn's lemma now implies that $\mathcal{R}$ has a maximal element, which can easily be shown to be a real closure of $F$. With some work, one sees that Theorem \ref{formallyreal-orderability} implies that this real closure is indeed real closed.
\end{proof}

\begin{example}
	\begin{itemize}
		\item The real closure of $\bQ$ is the field of real algebraic numbers $\bar{\bQ}\cap \bR$
		\item The real closure of the field of real Laurent series $\bR((T))$ is the field of Puiseux series $\R\langle\langle T\rangle\rangle$. 
		\item The real closure of the field of rational functions $\bR(T)$ is the field of algebraic Puiseux series, i.e. those Puiseux series which are algebraic over $\bR(T)$.
	\end{itemize}
\end{example}

Given a real closed field $R$, the unique ordering making $R$ into an ordered field is given by $x\leq y$ iff $\exists z (y-x=z^2)$. This simple fact, together with Theorem \ref{formallyreal-orderability} allows us to do something quite unexpected: we can axiomatize the theory of real closed fields in the language of rings! Let $\catname{RCF}$ be the $\mathcal{L}_r$-theory consisting of 
\begin{itemize}
	\item the field axioms,
	\item a demonstration that the field is formally real: for each $n\geq 1$, the sentence $$\forall x_1\dots \forall x_n(-1\neq x_1^2+\dots+x_n^2),$$
	\item an exhibition of a root for every odd-degree polynomial: for each $n\geq 1$, the sentence $$\forall a_0\dots \forall a_{2n} \exists x(x^{2n+1}+a_{2n}x^{2n}+\dots+a_1x+a_0=0),$$ and
	\item an exhibition of a square-root for every positive element: the sentence $\forall x\exists y( (y^2=x)\vee y^2=-x) )$
\end{itemize}

We may furthermore construct the theory $\catname{RCF}_{\leq}$ in the language $\mathcal{L}_{or}=\mathcal{L}\cup \{\leq\}$ which consists of $\catname{RCF}\cup\{\forall x\forall y (x\leq y \iff \exists z (y-x=z^2) ) \}$. These two theories have the same models, but it is sometimes convenient to have the extra symbol $\leq$. Thus the models of $\catname{RCF}$ (and hence also $\catname{RCF}_{\leq}$) are precisely the real closed fields.

There are many equivalent formulations of a real closed field. The most common are given in the following

\begin{theorem}
\label{artin-schreier}
Let $R$ be a field. The following are equivalent:
	\begin{enumerate}
		\item $R$ is real closed
		\item $R$ is formally real and has no proper formally real algebraic extensions
		\item $R$ is not algebraically closed, but $R(\sqrt{-1})$ is algebraically closed
		\item The absolute Galois group of $R$ is finite and non-trivial (and as a consequence of condition 3 it must be a cyclic group of order 2)
		\item $R$ is elementary equivalent to $\bR$, i.e. given any $\mathcal{L}_r$-sentence $\phi$, $R\models\phi$ iff $\bR\models \phi$
	\end{enumerate}
\end{theorem}
\begin{proof}
The equivalence of conditions (1)-(4) is the famous Artin-Schreier theorem \cite{artin1927algebraische}. The equivalence of (1) and (5) follows from the fact that $\catname{RCF}$ and $\catname{RCF}_{\leq}$ are model complete \cite{real}[Proposition 5.2.3]
\end{proof}

Of course, it is condition 5 which is of interest to us. To demonstrate the power of this theorem, we can now make short work of Hilbert's proposed problem: 
\begin{theorem}
\label{hilberts17th}
Let $R$ be a real closed field and let $f\in R[x_1,\dots, x_n]$ be such that $f(a_1,\dots, a_n)\geq 0$ for all $(a_1,\dots, a_n)\in R$. Then $f$ is a sum of squares in $R(x_1,\dots, x_n)$.
\end{theorem}
\begin{proof}
To avoid confusion, let us differentiate the polynomial \emph{form} $f\in R[x_1,\dots, x_n]$ from the polynomial \emph{function} $\hat{f}:R^n\ra R$ determined by $f$.

\noindent Suppose $f$ is not a sum of square in $R(x_1,\dots, x_n)$. Since $R$ is formally real, so is $R(x_1,\dots, x_n)$. By Theorem \ref{formallyreal-orderability}, we may choose an ordering on $R(x_1,\dots, x_n)$ so that $f<0$. Let $\mathcal{R}$ be a real closure of $R(x_1,\dots, x_n)$. Then $\mathcal{R}\models \exists a_1,\dots, a_n (\hat{f}(a_1,\dots, a_n)<0)$. Indeed, taking $a_i=x_i$ we get $\hat{f}(a_1,\dots, a_n)=f<0$. 
By the model-completeness of real closed fields, we find that $R\models \exists a_1,\dots, a_n (\hat{f}(a_1,\dots, a_n)<0)$ as well.
\end{proof}

As we've established a link with model theory, it is worth investigating the definable subsets in $\catname{RCF}_{\leq}$. Now given access to the ordering $\leq$, it is natural to extend the notion of an algebraic set as it is found in classical algebraic geometry to that of a \emph{semialgebraic} set:

\begin{definition}
Let $R$ be a real closed field. A subset $A\subset R^n$ is called \emph{semialgebraic} if it is definable in $\mathcal{L}_{or}$ by a finite boolean combination of polynomial equalities and inequalities.
\end{definition}

\begin{example}
	\begin{itemize}
		\item The subset of $\bR^2$ defined by $ xy=1\wedge x>0$ is semialgebraic. It's projection onto the $x$-axis is semialgebraic, but not algebraic.
		\item The subset of $\bR^2$ defined by $(x^2+y^2-1)^3\leq x^2y^3$ makes a rather lovely semialgebraic set.
		\item Every semialgebraic subset of $\bR$ is a finite union of points and intervals. This property (called \emph{o-minimality}) turns out to be wildly useful. A detailed analysis of o-minimal expansions $\bR$ is the cornerstone of the aforementioned proof of the Andr\'e-Oort conjecture \cite{pila2022canonical}.
		\item The subset of $\bR^2$ defined by $y=\lfloor x \rfloor$ is not semialgebraic.
	\end{itemize}
\end{example}

The keen reader might implore as to why we have seemingly restricted ourselves to boolean combinations of polynomial (in)equalities. \emph{What of quantifiers?} they bravely ask. But 
\emph{nay} sayeth Tarski, a man with little time for such trifling things:

\begin{theorem}
\label{tarski-seidenberg}
$\catname{RCF}_{\leq}$ admits \emph{quantifier elimination}, i.e. given any real closed field $R$ and $\mathcal{L}_{or}$-formula $\phi(x_1,\dots, x_n)$, there is a quantifier-free $\mathcal{L}_{or}$-formula $\psi(x_1,\dots, x_n)$ so that for all $(a_1,\dots, a_n)\in R^n, R\models \phi(a_1,\dots, a_n)$ iff $R\models \psi(a_1,\dots, a_n)$ (that is to say, $\phi$ and $\psi$ determine the same definable subset in all models of $\catname{RCF}_{\leq}$). Put geometrically, every $\mathcal{L}_{or}$-definable subset of $R^n$ is semialgebraic.
\end{theorem}
\begin{proof}
This is the Tarski-Seidenberg theorem \cite{real}[Theorem 1.4.2].
\end{proof}

If $R$ is real closed, we can define the \emph{order topology} on $R$ by the declaring the basic open subsets to be the open intervals $(a,b):=\{c\in R: a<c<b\}$ with $a,b\in R$; the product topology induced on $R^n$ (generated by products of open intervals) is also called the order topology. Real closed fields with the order topology possess many of the nice topological and analytic properties of $\bR$ with the Euclidean topology, provided we restrict ourselves to definable sets and definable functions: the least upper bound property, intermediate value theorem, limits and derivatives, inverse function theorem, implicit function theorem etc. These results (and many more) can be found in \cite{real}.

\section{Abelian varieties over $\bR$}
Our goal is to use the transfer principle, Theorem \ref{transfer-theorem}, to obtain results about abelian varieties over arbitrary real closed fields using known results for those defined over $\bR$.
The result of this section can be found in \cite{gross-harris}. The proof is real analytic in nature and is reproduced here to illustrate that Weil's philosophy of the Lefschetz principle holds equally well in the setting of real closed fields.

\begin{theorem}
\label{structure-of-real-AVs}
Let $A/\bR$ be an abelian variety of dimension $g$ defined over $\bR$. Then there is an integer $d$ with $0\leq d\leq g$ so that $$A(\bR)\isom (\bR/\bZ)^g \times (\bZ/2\bZ)^d$$
\end{theorem}
\begin{proof}
Let $A(\bR)^0$ be the connected component of the identity in $A(\bR)$. Then since $A(\bR)^0$ is a compact, connected, abelian, real Lie group of dimension $g$, it is a torus: $A(\bR)^0\isom (\bR/\bZ)^g$ (see \cite{compactliegroups}, for instance). Since this group is divisible, the exact sequence $$0\ra A(\bR)^0 \ra A(\bR)\ra A(\bR)/A(\bR)^0\ra 0$$ splits, and we find that $A(\bR)\isom (\bR/\bZ)^g\times A(\bR)/A(\bR)^0$. Now it suffices to demonstrate that the component group $A(\bR)/A(\bR)^0$ is a 2-group whose rank falls in the specified range.

Consider the norm map $N:A(\bC)\ra A(\bR)$ given by $N(P)=P+\bar{P}$. $N$ is a continuous homomorphism and $A(\bC)$ is compact and connected, so the image $N(A(\bC))$ is a closed, connected subgroup of the compact group $A(\bR)$ containing the identity. On the other hand, the image $N(A(\bC))$ contains the finite index subgroup $2A(\bR)$ and so is itself finite index and therefore open. We conclude that $N(A(\bC))=A(\bR)^0$ and furthermore that $A(\bR)/A(\bR)^0$ is killed by 2. $A(\bR)$ is compact, so the component group is finite. Thus $A(\bR)\isom (\bR/\bZ)^g\times (\bZ/2\bZ)^d$ for some $d$. The bound on $d$ follows from Proposition \ref{torsion-C}
\end{proof}

As we have no hope of giving a meaningful interpretation to the torus $\bR/\bZ$ over an arbitrary real closed field, we here remark that $\bR/\bZ$ is uniquely divisible and torsion-split, i.e. $\bR/\bZ\isom V\times \bQ/\bZ$, where $V$ is a $\bQ$-vector space of cardinality of the continuum.

\section{Abelian varieties over real closed fields}
Much of the content in this section can be found in the author's previous work \cite{lowry2023abelian}, but is here reformulated using the machinery developed in Section 1.6.

As mentioned above (Theorem \ref{frey-jarden}), Frey and Jarden show in \cite{frey-jarden} that for every abelian variety of dimension $g>0$ defined over an algebraically closed field $K$ of characteristic zero, there is an isomorphism of groups $$A(K)\cong V\times(\bQ/\bZ)^{2g},$$ where $V$ is a $\bQ$-vector space of dimension equal to $|K|$, the cardinality of $K$. We prove an analogous result for real closed fields:

\begin{theorem}
\label{structure-of-realclosed-AVs}
Let $R$ be a real closed field and $A/R$ an abelian variety of dimension $g>0$. Then there is an isomorphism of groups $$A(R)\cong V \times (\bQ/\bZ)^g\times (\bZ/2\bZ)^d$$ for some $0\leq g \leq d$, where $V$ is a $\bQ$-vector space of dimension $|R|$, the cardinality of $R$.
\end{theorem}

We first prove that $A(R)$ has the desired rank $\dim_{\bQ}A\otimes_{\bZ} \bQ$.

\begin{proposition}
\label{prop-freepart}
Let $A/R$ be an abelian variety of dimension $g>0$ over a real closed field. Then the rank of $A(R)$ is equal to the cardinality of $R$.
\end{proposition}
\begin{proof}
It is known that real closed fields are ``ample" (or ``large" as in Pop \cite{pop}) in the sense of Definition \ref{ample} above: certainly \cite{real}[Proposition 2.9.10] implies this claim; for a more down-to-earth demonstration, one could modify the proof given below to suit the task, for this fact is, at bottom, another consequence of the implicit function theorem. 

\indent As $R$ is ample, Theorem \ref{fehm-petersen} implies that $A(R)$ has infinite rank. If $R$ countable, then so too is $A(R)$ and the result follows. If, on the other hand, $R$ is uncountable, then one may employ the implicit function theorem for real closed fields (as in \cite{real}, Corollary 2.9.8) to find at least $|R|$ points in $A(R)$. Indeed, consider an affine open $U=\Spec(R[x_1,\dots, x_n]/(f_1,\dots, f_m))$ around the identity $e\in A(R)$. As $A$ is smooth, it is a local complete intersection, so we may assume that $n-m=g$. Consider the map $R^n\ra R^m$ given by the $f_i$. Again since $A$ is smooth, the Jacobian matrix $(\partial f_i / \partial x_j)$ has maximal rank $m$ at $e$ and thus there is some $m\times m$ minor of this matrix which is non-vanishing. After relabeling coordinates we may assume that $e$ belongs to the open set where $\det( (\partial f_i / \partial x_j)_{g+1\leq j\leq n})\neq 0$. Now the implicit function theorem guarantees the existence of a non-empty open subset $C\subset R^g$ and a definable function $h:C\ra R^m$ so that $(a_1,\dots, a_g, h(a_1,\dots, a_g))\in A(R)$ for every $(a_1,\dots, a_g)\in C$. As non-empty open subsets in $R^g$ have cardinality $|R|$, this demonstrates at least $|R|$ points of $A(R)$. As the cardinality of $A(R)$ is certainly at most that of $R$, we have $|A(R)|=|R|$. \newline \indent The torsion subgroup $T\subset A(R)$ is at most countable, so $|A(R)/T|=|R|$, since we assume $R$ to be uncountable. Finally, $A(R)/T$ is isomorphic to a subgroup of $A(R)\otimes \bQ$, so $|A(R)\otimes \bQ|=|R|$ as well. The result follows from a simple fact of cardinal arithmetic: if $V$ is a $\bQ$-vector space of uncountable cardinality $\kappa$, then $\dim_{\bQ} V=\kappa$.
\end{proof}

\begin{proposition}
\label{torsion-prop-real-closed}
Let $R$ be a real closed field, $A/R$ an abelian variety of dimension $g>0$. Then $A(R)[\tors]\isom (\bQ/\bZ)^g\times (\bZ/2\bZ)^d$ for some $0\leq d\leq g$.
\end{proposition}
\begin{proof}
Arguing exactly as in the proof of Proposition \ref{torsion-alg-closed-transfer}, there is an abelian scheme $\mathcal{A}/S$ with $S$ affine and finite type over $\Spec(\bZ)$ together with an $R$-point $s:\Spec(R)\ra S$ such that the fiber $\mathcal{A}_s/R$ is isomorphic to $A/R$ as an abelian variety over $R$. Now for fixed odd $p$ and fixed natural numbers $n$ and $g$, the statement ``there are exactly $p^{ng}$ elements killed by $p^n$" is a first-order statement in the language of groups which, by Theorem \ref{structure-of-real-AVs}, is known to hold for the fiber over each $\bR$-point of $S$. By Theorem \ref{transfer-theorem}, it must hold for $A$ as well. 

Similarly, the sentence ``for some $0\leq d\leq g$, there are exactly $2^{g+d}$ elements killed by 2" is a first-order sentence in the language of groups which, again by Theorem \ref{structure-of-real-AVs}, is known to hold for the fiber over each $\bR$-point of $S$, and so by Theorem \ref{transfer-theorem}, must hold for $A$ as well. Now, for each $n$, the first-order sentence ``for every $0\leq g\leq d$, if there are exactly $2^{ng+d}$ elements killed by $2^n$, then there are exactly $2^{(n+1)g+d}$ elements killed by $2^{n+1}$" is known to hold for the fiber over each $\bR$-point of $S$ and so again by Theorem \ref{transfer-theorem}, these statements holds for $A$. This sequence of statements amounts to the claim that there is some \emph{fixed} $0\leq d \leq n$ so that for each $n$, $|A[2^n](R)|=2^{ng+d}$.

As in the case of an algebraically closed field, purely group-theoretic considerations now imply that the torsion subgroup of $A$ has the desired form. Indeed, every torsion group is a direct sum of its $p$-power torsion subgroups and the counts given above, together with the fundamental theorem of finite abelian groups, give $A(R)[\tors]$ the desired form.
\end{proof}

NOTE: I think we came up with a proof of this fact that doesn't require the model theory, at least when $A$ is principally polarized. It was something like ``look at the action of conjugation on the $\bZ/p\bZ$-vector space $A[p](R(\sqrt{-1}))$ and use the Weil pairing." Should we include this?

We are now ready to prove Theorem \ref{structure-of-realclosed-AVs}:

\begin{proof}
The above proposition shows that the torsion subgroup $T$ of $A$ is a divisible group times a finite group; it follows from \cite[Theorem 8.1]{baer} that $A(R)$ is torsion-split, i.e. $A(R)\cong V\oplus T$, where $V$ is torsion free. We need only to verify that $V$ is uniquely divisible. First note that $2A(R)$ is uniquely divisible. Indeed, it is uniquely $n$-divisible for each $n$, as this is a first-order statement in the language of groups which can be verified over the $\bR$-points in a definable family containing $A$. Now given any $v\in V$, $2v\in 2A(R)$ whence $2v/2n\in A(R)$, i.e. there is some $y\in A(R)$ so that $2v=2ny$. Writing $y=w+t$ for $w\in V$ and $t\in T$, we find that $2v=2nw$, and since $V$ is torsion-free we must have $v=nw$. That $V$ is \emph{uniquely} divisible follows since $2A(R)$ is uniquely divisible. Finally, that $V$ has the given dimension follows from Proposition \ref{prop-freepart}.
\end{proof}

\begin{corollary}
Let $R$ be a real closed field, $p$ an odd prime, and $A/R$ an abelian variety of dimension $g$. Then the Tate module $T_p(A)$ is a free $\bZ_p[G]$-module of rank $g$ where $G=\Gal(R(\sqrt{-1})/R)$.
\end{corollary}
\begin{proof}
As $A$ is defined over $R$, $T_p(A)$ comes equipped with an action of $\Gal(R(\sqrt{-1})/R)=:G$. Let us denote the action of the nontrivial element $\sigma\in G$ on $T_p(A)$ by $\overline{P}:=\sigma(P)$ (see Theorem \ref{artin-schreier}, condition 4). It suffices to produce $P_1,\dots P_g, Q_1,\dots Q_g\in T_p(A)$ with $\overline{P_i}=P_i$ and $\overline{Q_j}=-Q_j$ for $1\leq i,j\leq g$ such that $$T_p(A)=\bigoplus_{i=1}^g (\bZ_p\cdot P_i)\oplus\bigoplus_{j=1}^g (\bZ_p\cdot Q_j).$$ Indeed, for fixed $i=1,\dots,g$ the $\bZ_p[\bZ/2\bZ]$-module $\bZ_p\cdot P_i\oplus\bZ_p \cdot Q_i$ is free with basis $\{P_i+Q_i,P_i-Q_i\}$ --- notice here we crucially use the fact that $p\neq 2$: the matrix 
$$\begin{bmatrix} 
1 & 1\\
1 & -1
\end{bmatrix}$$
is invertible over $\bZ_p$, but not over $\bZ_2$! We construct the $P_i$ and $Q_j$ by lifting inductively from $\bZ/p\bZ$. 

$\sigma$ is an involution on the $2g$-dimensional $\bZ/p\bZ$-vector space $A[p]$. $\sigma$ is diagonalizable, again using the fact that $p\neq 2$; we thus split $A[p]$ into a $+1$- and $-1$-eigenspace, identifying the former with $A[p](R)$. By Theorem \ref{structure-of-realclosed-AVs}, each is of dimension $g$ (over $\bZ/p\bZ)$, so $A[p]$ has a $\bZ/p\bZ$-basis $P_{1,1},\dots, P_{1,g},Q_{1,1},\dots, Q_{1,g}$ with $\overline{P_{1,i}}=P_{1,i}$ and $\overline{Q_{1,j}}=-Q_{1,j}$; this is the desired structure, modulo $p$. Suppose now we have the desired decomposition $$A[p^n]=\bigoplus_{i=1}^g(\bZ/p^n\bZ)\cdot P_{n,i}\oplus \bigoplus_{j=1}^g(\bZ/p^n\bZ)\cdot Q_{n,j}$$ for $n\geq 1$. Fix $1\leq i\leq g$. As the multiplication-by-$p$ morphism has degree $p^{2g}$ (Proposition \ref{torsion-alg-closed}), $[p]^{-1}(P_{n,i})$ has $p^{2g}$ elements. Moreover, $[p]^{-1}(P_{n,i})$ is a $G$-set with the obvious action of $\sigma$. Since $[p]^{-1}(P_{n,i})$ is partitioned by the $G$-orbits and $p^{2g}$ is odd (\emph{again} using that $p\neq 2$), some element must have an orbit of size 1, i.e. there is some $P_{n+1,i}\in A[p^{n+1}](R)$ with $[p](P_{n+1,i})=P_{n,i}$. Essentially the same argument works to produce $Q_{n+1,j}$ with $[p](Q_{n+1,j})=Q_{n,j}$ and $\overline{Q_{n+1,j}}=-Q_{n+1,j}$, but here we have $\sigma$ act on $[p]^{-1}(Q_{n,j})$ with a twist by $-1$. (NOTE: am i using the word "twist correctly"? I mean that if $S\in [p]^{-1}(Q_{n,j})$, then $\sigma\cdot S:=-\sigma(S)$. Also, this is just the action of $G$ on the ``twisted" abelian variety you get via Galois descent; should I mention this?) That the $P_{n+1,i}$ and $Q_{n+1,j}$ are $\bZ/p^{n+1}\bZ$-linearly independent and span $A[p^{n+1}]$ follows immediately. In this way we form a compatible system of representatives so that the $P_i=(P_{n,i})_{n=1}^{\infty}$ and $Q_j=(Q_{n,j})_{n=1}^{\infty}$ have the desired properties.
%we can probably figure out the possibilities for Z_2: spilt into cases 1. the min poly mod 2 is (x-1), 2. the min poly mod 2 is (x-1)^2. does our lifting strategy work just as well? maybe this is where you get some Q_2*/(Q_2^*)^2 stuff
\end{proof}

The final result of this section examines how the isomorphism type of the group of rational points of a real abelian variety varies in a family. The structure theorem proved above demonstrates that this isomorphism type is entirely determined by the number of 2-torsion points, indicating a possible avenue of approach. As an illustrative example, let us return again to our favorite family $\mathcal{E}\ra \Spec(\bZ[a,b][1/(4a^3+27b^2)])$. One notices that the elliptic curves with a \emph{connected} real locus occur over the fibers with $4a^3+27b^2>0$; such curves have two real 2-torsion points and, as real Lie groups, each is isomorphic to the 1-dimensional torus $\bR/\bZ$. Those elliptic curves with $4a^3+27b^2<0$ have a \emph{disconnected} real locus; they have four real 2-torsion points and each is isomorphic to $\bR/\bZ\times\bZ/2\bZ$ as a real Lie group.

\begin{proposition}
\label{deformation}
Let $\mathcal{A}\ra S$ be an abelian scheme of dimension $g$ with $S$ of finite type over a real closed field $R$. Then the function $S(R)\ra \bN$ sending $s\in S(R)$ to $|\mathcal{A}_s[2](R)|$ is locally constant with respect to the order topology on $S(R)$ inherited from the ordering on $R$. In particular, the isomorphism class of the fiber (as an abelian group) is constant on the connected components of $S(R)$.
\end{proposition}
\begin{proof}
We seek to apply a suitable version of the following elementary fact from the theory of real manifolds, whose proof is a simple application of the inverse function theorem: let $f:M\ra N$ be a smooth map between real manifolds of the same dimension with $M$ compact; if $U$ is the (open) subset of $N$ consisting of regular values, then the function $U\rightarrow \bN$ which sends $y\in U$ to $|f\inv(y)|$ is locally constant, ``i.e. there is a neighborhood $V\subset N$ of $y$ such that $|f\inv(y')|=|f\inv(y)|$ for any $y'\in V$. [Indeed,] let $x_1,\dots, x_k$ be the points of $f\inv(y)$, and choose pairwise disjoint neighborhoods $U_1,\dots, U_k$ of these which are mapped diffeomorphically onto neighborhoods $V_1,\dots, V_k$ in $N$. We may then take $V=V_1\cap V_2\cap \cdots \cap V_k- f(M- U_1-\cdots- U_k)."$ \cite[pg. 8]{milnor}

An \'etale morphism of varieties is the algebraic analogue of a local diffeomorphism. The inverse function theorem fails in the coarse Zariski topology, but as it holds for the order topology on a real closed field, it should come as no surprise that an \'etale morphism of varieties $f:V\ra W$ induces a local diffeomorphism of real algebraic sets $f:V(R)\ra W(R)$ \cite[Proposition 8.1.2]{real}.

Now, the 2-torsion subgroup scheme $\mathcal{A}[2]$ is \'etale over $S$, as it is the pullback of the \'etale morphism $[2]:\mathcal{A}\ra \mathcal{A}$. At the level of $R$-points, we have local diffeomorphism $\mathcal{A}[2](R)\ra S(R)$. The fiber over each point of the base if finite and non-empty by Theorem \ref{structure-of-realclosed-AVs}, so the exact same proof given in the first paragraph yields that the number of 2-torsion points in the fiber over each point of the base $S(R)$ is locally constant. The isomorphism class of a real abelian variety (as an abelian group) is completely determined by its 2-torsion subgroup (Theorem \ref{structure-of-realclosed-AVs}), so the second statement follows.
\end{proof}
\chapter{Abelian varieties over $p$-adically closed fields}
asdf

\section{Abelian varieties over $\bQ_p$}
asdf

\section{Abelian varieties over arbitrary $p$-adically closed fields}
asdf
\cite[Exercise 2.1.20(a)]{BrunsHerzog-CM_rings}

% \chapter{Documentation}

% This document doubles as documentation for the \texttt{iuphd.cls} and as a sample for the output. The \texttt{iuphd} class is based on the standard \LaTeX $~$ report class, and as such you may assume that it follows the default formatting of the report class unless otherwise specified below. The changes to the report class include the following:

% \begin{itemize}
% \item changes to the default settings for margins, font size (11pt), spacing (double), and pagination,
%  \item the quote, quotation, and verse environments have been changed to switch temporarily to single spacing, and
%        remove the extra space at the top and bottom,
%  \item the enumerate and itemize environments have been changed to remove extra space between the items and at the
%        top and bottom,
%  \item a line has been added to prevent the double spacing from affecting arrays,
%  \item both numbered and unnumbered chapter headings are now centered by default,
%  \item changes to the definitions of \textbackslash maketitle and the `abstract' environment,
%  \item the addition of commands for creating the acceptance page and copyright page,
%  \item the addition of the `acknowledgments', `preface', and `dedication' environments,
%  \item the addition of macros for specifying the department or school name, the names of
%  the committee members, the defense date, and the copyright year,
%  \item document class options for changing ``Department'' to ``School'' on the title page, and for showing
%  the abstract, preface, acknowledgments, or dedication in the table of contents.
% \end{itemize}

% \newpage

% The latest update (Summer 2018) makes the following changes from the last update (Fall 2013):
% \begin{itemize}
% \item Makes the left and right margins a uniform width of 1" on all pages.
% \item Headings have been changed so that they are no longer forced to be uppercase and so that they match the font size requirements of the Graduate School.
% \item Spacing of chapter headings has been changed so that the chapter headings appear at the top of the page.
% \item Lines have been added for the signatures on the Acceptance and Abstract pages.
% \item The Dedication page has been changed to remove the title and to center the text both horizontally and vertically.
% \end{itemize}


% The appropriate commands for each section are detailed in the section that follows. You can comment out any section you do not want by using a $\%$.

% \section{Current Section Standards}
% The following document \emph{should} adhere to the current standards of the Graduate School (last updated in Summer 2018). The current order of the sections must be

% \begin{itemize}

% \item Title Page
% \item Acknowledgements (optional)
% \item Dedication (optional)
% \item Preface (optional)
% \item Abstract
% \item Dissertation Body
% \item Bibliography
% \item Appendices, Indexes, List of Figures/Tables (optional)
% \item CV
% \end{itemize}

% Note: Some styles guides prefer appendices to come before the bibliography. Check your field's standards on this.


% \subsection{Table of Contents}

% Sections coming before the dissertation body appear in the table of contents without a chapter number and are numbered with Roman numerals. There does not seem to be a standard about whether the abstract, preface, acknowledgments, and dedication are listed in the table of contents (TOC) or not.  As such, an option has been provided for including or excluding each one individually.  The syntax for these options follows the pattern `showX' if X is to be listed in the TOC, and 'hideX' if X is to be omitted from the TOC, where X can be any of the four items listed above.  In all cases, the default is for these \emph{not} to have entries in the table of contents.  So for example, to include both the abstract and the preface in the table of contents, but not the dedication or acknowledgments, you would use the code:

% \medskip

% \textbackslash documentclass[showabstract,showpreface]\{iuphd\}
% \medskip

% These commands are in use for this document but can be changed as necessary. Depending on how you include the sections that come before the dissertation body, you might have a numbering issue. This can be remedied by adding the line \textbackslash{setcounter[page]{0}} immediately before the start of the first chapter of the dissertation body.

% \subsubsection{Including Sections in Bibliography}
% \noindent To include a line for the bibliography in the table of contents, the command

% \medskip

% \textbackslash addcontentsline\{toc\}\{chapter\}\{Bibliography\}
% \medskip

% \noindent should be placed directly above the \textbackslash bibliography and \textbackslash bibliographystyle commands.  It may be necessary to use \textbackslash clearpage above this command to get the page number displayed in the contents to be correct.  However, if you are using \textbackslash include for each chapter, then \textbackslash clearpage has technically already been used in this case, and so it is not necessary to use it again. Similar commands will work for including any unnumbered heading in the table of contents, with careful attention to the effects of \textbackslash clearpage.

% \subsection{CV}

% A CV is also required following the bibliography, however, to maintain maximum flexibility no special tools have been provided for formatting a CV.  Various packages are available that do this, although custom formatting is best. As such, only a line for including the CV after the bibliography is included. Please note that the Graduate School currently requires the CV to have the same font as the rest of the dissertation. 

% \section{Commands}

% Various commands have been defined for the data that will show up on the title page, acceptance page, and copyright page.  These include:
% \begin{itemize}
%  \item \textbackslash title - for the title of the dissertation
%  \item \textbackslash author - for the name of the author of the dissertation
%  \item \textbackslash date - for the \emph{completion date} of the dissertation
%  \item \textbackslash department - this is optional.  The default value is ``Mathematics,'' but if you are in a different department, this
%         command can be used to change its value to the name of whatever department (or school) you are in.  
%  \item \textbackslash committeechair, \textbackslash readertwo, \textbackslash readerthree, \textbackslash readerfour - for the names of the
%                       dissertation committee members.
%  \item \textbackslash defensedate - for the \emph{defense date} of the dissertation (which may not match the completion date).
%  \item \textbackslash cryear - for the copyright year.
% \end{itemize}

% Additionally the commands \textbackslash acceptancepage and \textbackslash copyrightpage have been defined to generate the acceptance page and copyright page, respectively.  The copyright page is not required, and is commented out in this sample document.  Two document class options have been provided for switching the text on the title page between ``Department'' and ``School.''  These options are given the obvious names `department' and `school'. The default option is `department', so it only needs to be changed if you are in a school rather than a department.

% \chapter{Suggestions for additional formatting}

% \section{The \textbackslash include feature}

% For longer documents, it is often easier to manage if it is broken into multiple files.  This can be done by using the \textbackslash include\{filename\} or \textbackslash input\{filename\} commands.  You can think of \textbackslash input\{filename\} as an alias for all of the text contained in filename.tex: the text contained in that file,
% simply gets inserted at the exact spot where the command \textbackslash input is used.  This means that filename.tex should not have \textbackslash begin\{document\}, \textbackslash end\{document\}, or a preamble. On the other hand, \textbackslash include\{filename\} has the additional feature of adding \textbackslash clearpage both before and after the text being inserted. As an example

% \textbackslash chapter\{Introduction\}

% \textbackslash include\{chap1\}
% \medskip

% gives the title of chapter 1 as Introduction and the body of the chapter is in the file \texttt{chap1.tex}.

% \section{Useful Packages}

% Many \LaTeX \ packages are already available for additional customization of the formatting which is otherwise built in.
% Some of these are listed below.  To use them, simply declare them with \textbackslash usepackage in the preamble
% and refer to the documentation files for them, which are easily available online.

% \begin{itemize}
%  \item titlesec - for custom formatting of part, chapter, and section headings.
%  \item nomencl - for lists of notation (a tweak of makeindex).
%  \item prettyref - for custom formatting of references (e.g. referencing equations as (1), (2), etc., instead of the default
%        1, 2, etc.).  The ams packages also have a similar feature but it is much more limited.
%  \item natbib, custom-bib, or biblatex - offer additional flexibility for bibliography formatting beyond the standard used by
%        the mathematics community.
%  \item setspace - contains features for resetting the global line spacing, as well as environments that can be used to change
%        spacing locally, which are compatible with the quote, quotation, and verse environments.
%  \item quoting - provides a more flexible quotation environment, with options for adjusting the margins manually.  This package
%      should be used with setspace, to ensure single spacing of the quotations.
%  \item enumitem or paralist - provide several more flexible list environments.
%  \item titling - for custom changes to maketitle.
%  \item geometry - for customizing the page layout.
% \end{itemize}

% %add features for block quotes, underlined titles, and different departments.

% The last two should not be necessary unless the standards at IU change.  However, if that happens, it would really be better to
% rewrite the iuphd document class.

% \section{Quote samples}
% The quote environment:
% \begin{quote}
% ``God does not play dice with the universe.'' - Albert Einstein
% \end{quote}
% The quotation environment:
% \begin{quotation}
%  ``A GREAT discovery solves a great problem, but there is a grain of discovery in the solution of any problem.
%  Your problem may be modest, but if it challenges your curiosity and brings into play your inventive faculties,
%  and if you solve it by your own means, you may experience the tension and enjoy the triumph of discovery.''
 
%  - George P\'olya
% \end{quotation}
% The verse environment:
% \begin{verse}
% Im September ist alles aus Gold:\\ 
% die Sonne, die durch das Blau hinrollt,\\
% das Stoppelfeld,\\
% die Sonnenblume, schläfrig am Zaun,\\
% das Kreuz auf der Kirche,\\
% der Apfel im Baum.\\
% Ob er h\"alt,\\
% ob er f\"allt? \\
% Da wirft ihn geschwind\\
% der Wind\\
% in die goldene Welt.

% - Georg Britting
% \end{verse}




\addcontentsline{toc}{chapter}{Bibliography}

\newpage

% \appendix command is necessary to change chapter numbering.
% Appendices are optional

%\appendix
%
%\input{Der_Cats_Intro.tex}
%\input{CM_Schemes.tex}
% \input{Base_loci.tex}
% \chapter{Troubleshooting}
\clearpage

% If you are getting an error with the TOC, you might have compiled a previous *.toc file with a different latex class (e.g. the amsbook class). To remedy this, delete the current *.toc file and recompile.

% \bigskip

% Did you remember to use \textbackslash addcontentsline and or
% \textbackslash clearpage commands where necessary?




% % Adds a line for your CV without a page number
\addtocontents{toc}{\cftpagenumbersoff{chapter}}
\chapter*{Curriculum Vitae}
\addcontentsline{toc}{chapter}{Curriculum Vitae}
\thispagestyle{empty}
\textbf{Basic Information}
\vspace{1pt}
\hrule 
\vspace{\baselineskip}
\noindent Name: Dylan Spence \\
Address: Rawles Hall 230, 831 East 3rd St., Bloomington IN 47405 \\
Website: \url{dkspence952.github.io}\\
Email: dkspence@indiana.edu

\noindent \textbf{Education}
\vspace{1pt}
\hrule 
\vspace{\baselineskip}

\noindent \textbf{Indiana University Bloomington} \hfill \emph{May 2022} \\
Ph.D. Mathematics \hfill Advisor: Valery Lunts \\
Thesis: \emph{Derived categories of singular curves}

\noindent \textbf{University of Delaware} \hfill \emph{May 2016} \\
B.S. Applied Mathematics and B.S. Physics 

\noindent \textbf{Research Interests}
\vspace{1pt}
\hrule 
\vspace{\baselineskip}

\noindent Algebraic geometry. In particular derived categories, deformation theory, moduli, and related equivariant questions. Applications to homological mirror symmetry, birational geometry, geometric representation theory, and motives. Further interests include noncommutative and higher algebraic structures arising in the above.

\noindent \textbf{Publications}
\vspace{1pt}
\hrule 
\vspace{\baselineskip}

\noindent \textbf{Research Publications}
\begin{enumerate}
      \item \emph{A note on semiorthogonal indecomposability of some Cohen-Macaulay varieties.} Journal of Pure and Applied Algebra, Volume 226, Issue 10, 2022. DOI: 10.1016/j.jpaa.2022.107076
      \item \emph{Reconstruction of Projective Curves from the Derived Category}. Michigan Math. J. Advance Publication 1 - 24, 2021. DOI: 10.1307/mmj/20205910
\end{enumerate}

\noindent \textbf{Other Publications}
\begin{enumerate}
      \item \emph{Boundedness of semistable sheaves.} with H. Guo, S. Shivaprasad, and Y. Wu. \href{https://arxiv.org/abs/2112.03834}{arXiv 2112.03834}. Submitted as part of the Stacks Project Expository Collection (SPEC) to be published in the London Mathematical Society Lecture Note Series.
\end{enumerate}

\thispagestyle{empty}

\noindent \textbf{Teaching Experience}
\vspace{1pt}
\hrule 
\vspace{\baselineskip}

\noindent\textbf{Indiana University Bloomington} \\
Instructor of Record:
\begin{enumerate}
      \item Math-D 116: Introduction to Finite Math I \hfill \emph{Spring 2022}
      \item Math-J 111: Introduction to College Math I \hfill \emph{Fall 2021}
      \item Math-M 106: Math of Decision and Beauty \hfill \emph{Fall 2020}
      \item Math-J 111: Introduction to College Math I \hfill \emph{Fall 2019}
      \item Math-M 106: Math of Decision and Beauty \hfill \emph{Summer 2019}
      \item Math-J 113: Introduction to Calculus with Applications \hfill \emph{Spring 2019}
      \item Math-J 010: Introduction to Algebra \hfill \emph{Summer 2018}
      \item Math-M 106: Math of Decision and Beauty \hfill \emph{Spring 2018}
      \item Math-M 106: Math of Decision and Beauty \hfill \emph{Summer 2017}
      \item Math-M 014: Basic Algebra \hfill \emph{Fall 2016}
\end{enumerate}

\noindent \textbf{Undergraduate Mentoring}
\vspace{1pt}
\hrule 
\vspace{\baselineskip}

\noindent \textbf{Directed Reading Program} \hfill Program Coordinator 2018 - 2022 \\
Graduate Mentor:
\begin{enumerate}
      \item Nathaniel Lowry. \hfill \emph{Spring 2019} \\ Project Title: Rational points on elliptic curves \\ Book: \emph{Elliptic curves} by Dale Husem\"oller. \vspace{1em}
      \item Nathanial Lowry. \hfill \emph{Spring 2018} \\ Project Title: Introduction to algebraic geometry  \\ Book: \emph{Algebraic Geometry: A Problem Solving Approach} by Thomas Garrity, et. al.
\end{enumerate}

\noindent \textbf{Laboratory of Geometry IU} \hfill Project Supervisor Spring 2022\\
Project: \emph{Lines on real cubic surfaces}

\thispagestyle{empty}
\noindent \textbf{Awards and Honors}
\vspace{1pt}
\hrule 
\vspace{\baselineskip}

\noindent \textbf{Hazel King Thompson Fellowship} \hfill Summer 2020

\noindent \textbf{David A. Rothrock Teaching Award} \hfill Spring 2018

\noindent \textbf{Hazel King Thompson Fellowship} \hfill Spring 2016

\noindent \textbf{Invited Talks}
\vspace{1pt}
\hrule 
\vspace{\baselineskip}

\noindent \textbf{Northwestern Algebraic Geometry Seminar} \hfill November 2021 \\
\emph{Derived categories of singular curves}

\noindent \textbf{Midwest Algebraic Geometry Graduate Conference} \hfill May 2020 \\
\emph{Reconstruction of projective curves from the derived category}

\noindent \textbf{University of Delaware Particle Physics Seminar} \hfill March 2018 \\
\emph{Complex and algebraic geometry in physics}

\noindent \textbf{Service and Professional Development}
\vspace{1pt}
\hrule 
\vspace{\baselineskip}

\noindent \textbf{IU Directed Reading Program Coordinator} \hfill 2018-2022 

\noindent \textbf{American Mathematical Society Member} \hfill 2016-Present

\noindent \textbf{Jumpstart: Algebra Instructor} \hfill Summer 2018, 2021 

\end{document}